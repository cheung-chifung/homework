\documentclass{jsarticle}
\topmargin=-3.0cm \oddsidemargin=0.1cm \evensidemargin=0.1cm
\textwidth=16 true cm \textheight=27 true cm
\setlength{\textheight}{250mm}

\usepackage{amsmath, amssymb}
\usepackage{fancybox,ascmac}

\begin{document}
\title{2012年度・価値創造の最前線・最終レポート}
\author{{\normalsize 社会理工学研究科 経営工学専攻 チョウ シホウ 学籍番号 12M42340}}
\date{}
\maketitle

\def \Pr{{\rm Pr}}


\baselineskip 1.5cm

\fontsize{5mm}{9mm}\selectfont

\section{価値創造とは?}
「価値創造の最前線から学んだこと」と問われたら,最初頭に浮び出すのは「価値創造」の言葉の意味だ.価値創造という言葉はよく耳にするが,今まで「利潤を作る」にすぎないと思った.しかし,授業で各分野で活躍している魅力的な講師の話を伺い,「価値創造」とは会社の利益を作ることだけではなく,社会への貢献,世界中の人々に幸せをもたらすまで深い意味を含んでいることは初めて知った.

ここでは,「価値創造の最前線」の授業からもっとも印象的な講座を選んで自分の感想と学んだことを話す.

\section{ベネッセコーポレーションの価値創造}
工業革命以来の長い間,人類の価値創造と言えば,新たな技術を発明したり,斬新な機械を製造したり,生産力の進化を極め,物質的財貨を生み出されることだろう.しかし,いかに努力しても,人類の生産力が頂上がある,頂上に達しても,世の中の人々は必ずしも幸せに成れるとは言えない.人間に真の幸福をもたらす為,「事業企業」が生み出した.

日本は世界に誇る事業企業を有する.戦後からの日本企業は,自然資源に限られ,今までのない事業を作らなければいけないため,新しい道を切り開いた.その時代の日本企業は,国民の暮らしのニーズを考え込み,炊飯器からインスタントラーメンまで幅広く画期的な商品を作り出し,世界のみんなさんの生活の質を大幅に改善しつつ,経済力も伸びした.いわゆる,その時代の価値を創造した.

価値創造の意味は時代とともに進化するもの.情報社会での価値創造とは,物質的なものづくりだけではなく,ローカルでグローバルの人々にサービスを提供し,価値を与えるのも可能になった.Googleなどのインターネットサービスを無料に提供して利益を得るのも,Appleなどの新商品を生み出して世界中の人に今までのない体験を与えるのも,情報時代の価値創造の一種と言っても過言ではない.その中,ベネッセのような人々の教育・暮らし・介護などの分野に参入し,国民生活の向上と伴い,新しい価値を生む価値創造もあった.


ベネッセコーボレーションの方々から学んだのは,ベネッセの事業内容と会社概要はもちろん,もっとも大事のは会社の理念と会社経営の関係をきちんと把握すること.会社はNGOではなく,経営活動を通じて利益を追求する組織である.会社ごとに理念と経営を同時に全うするのはたくさんがあるが,ベネッセの価値創造は社会も会社も有利なユニークな事業を行うこと,その方針で教育から介護までいろいろなサービスやプロダクトを提供しつつ,世界的経済不況で大きく利益をとり,会社の社会責任と経営活動を両立された.ベネッセコーポレーションの講座を通じて分かったのは,事業を初めから立つ際,予め事業の収益性,新規性,実現可能性,社会への貢献を統合して考えなければならないこと.

\section{Googleの価値創造}

ベネッセの価値創造と比べ,Googleのほうが技術の壁を突破しなければならないことが多い.そのため,Googleの価値創造は従来の物作り産業とは多少類似点がある.私もアプリ開発やWeb開発を行う経験があり,Googleはどうやって新たな技術とアイデアを生かして新しいものを作るのか知りたかった.この授業を通じて並んだのは,Googleの新技術や新アイデアは決して簡単に生み出せるものではなく,アイデアの提出・実装・具現化のプロセスはかなり工夫したものである.新しいアイデアが提出される際,すぐにその実現可能性を検討することではなく,やりがいがあれば技術面を突破すればよいという考えもある.

もちろん,技術の困難を克服しようでもできない可能性があるが,そうしないと従来のもとを作り繰り返し,価値創造とは言えない.また,Googleのような会社では,常に効率のよいプロジェクトインキューベータを持つ,新しいアイデアと技術を育てるし,一定な活性と成功率も保証できる.ITベンチャーならば,エスキュービズムのようにインターン生の作り試しに励み,ベンチャーの活性を持つ企業もある.

\section{世界銀行の価値創造}
他の企業の価値創造と比べ,世界銀行は極めて純粋な社会へ貢献する組織と見なせるが,他のNGOと比べると,世界銀行は各出資国の資金を利用し,発展途上国に長期資金を供給する,企業のような価値創造を行っている.この講座で谷口さんが世界経済のトレンドを分析し,今後はどうやって発展途上国の経済を促進するのかも論じた.差別せずに国境を越え,経済面で各困難に陥った国々に助かり,世界銀行はユニークな価値を創造したと思う.

\section{まとめ}
「価値創造」という概念は,人々の価値観,宗教,経歴によって十人十色だろう,しかし,最も重要なのは,各分野の会社や個人も自分なりに価値を作り出し,他のにその価値と経験を共有し,Win-Winに達すること.この授業では幅広い分野の最前線に立っている価値創造のパイオニア達の経験を伺え,有意義な授業と思う.


\end{document}
