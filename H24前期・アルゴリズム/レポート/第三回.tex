\documentclass[a4paper,11pt]{jsarticle}
\usepackage{graphicx}
\usepackage{wrapfig}
\usepackage{amsmath,amssymb,amsthm}
\usepackage{amssymb}
\usepackage{ascmac}
\usepackage{subfigure}
\usepackage{bm}
\usepackage{setspace}
\usepackage{cases}%左かっこつけるときに必要だった
\usepackage{leftidx}%行列表示用?
\usepackage{fancyhdr}
\usepackage{graphicx}
\usepackage{float}
\usepackage{booktabs}
\usepackage{url}
\usepackage{bm}
\usepackage{verbatim}
\usepackage{calc} 

\setlength{\headsep}{5mm}
\setlength{\oddsidemargin}{-0.5zw} %→にズラす
\setlength{\textheight}{37\baselineskip}
\addtolength{\textheight}{\topskip}
\setlength{\topmargin}{-10mm}
\setlength{\textwidth}{45zw} %文章の幅
\setlength{\textheight}{215mm}
\setlength{\parindent}{1zw}%箇条書きの一文字下げ
\pagestyle{fancy}

\newtheorem{theorem}{定理}
\newtheorem{prop}[theorem]{命題}
\newtheorem{lemma}[theorem]{補題}
\newtheorem{cor}[theorem]{系}
\newtheorem{example}[theorem]{例}
\newtheorem{definition}[theorem]{定義}
\newtheorem{rem}[theorem]{注意}
\newtheorem{guide}[theorem]{参考}
\renewcommand{\proofname}{証明}

\numberwithin{theorem}{section}  % 定理番号を「定理2.3」のように印刷
\numberwithin{equation}{section} % 式番号を「(3.5)」のように印刷
\newcommand{\argmax}{\mathop{\rm arg~max\,}\limits}
\newcommand{\argmin}{\mathop{\rm arg~min\,}\limits}
\newcommand{\st}{\mathop{\rm subject~to\,}\limits}
\newcommand{\sign}{\mathop{\rm sign\,}\limits}

\newcommand{\dom}{\mathop{\mathrm{\bf dom}}\nolimits}
\newcommand{\minimize}{\mathop{\mathrm{\rm minimize}}\limits}
\newcommand{\maximize}{\mathop{\mathrm{\rm maximize}}\limits}

\lhead{2012年度計算数理応用ーアルゴリズム・第三回レポート}
\chead{}
\rhead{12M42340}
\lfoot{チョウ シホウ}
\cfoot{\thepage}
\rfoot{6月11日}
\renewcommand{\footrulewidth}{0.4pt}
\title{2012年度計算数理応用ーアルゴリズム・第三回レポート}
\author{12M42340 チョウ シホウ}
\date{6月11日}

\begin{document}
\pagenumbering{roman}
%\tableofcontents
%\cleardoublepage
\pagenumbering{arabic}
\renewcommand{\thepart}{\arabic{part}}

%\thispagestyle{empty}
\begin{itembox}[l]{テーマ2・問1}
次の定理を証明せよ.

与えられた$n$に対して,一様ランダムに$(3,4)$-疎行列$H$を生成することを仮定する.また(あらかじめ決められている)パラメータ$p<<1$に対して,ノイズベクトル$\bm{n}_*\in\{0,1\}^n$を各ビットごと独立に確率$p$で$1$に,$1-p$で$0$となるように生成する.こうしたランダムモデルのもとで
\[
P_{H,\bm{n}_*}[\bm{n}_*\text{が}\bm{c}=G(\bm{n}_*,H)\text{のただ一つの最尤解である}]
\]
という確率が$n$を大きくしたときに$0$に収束する.
\end{itembox}
\begin{proof}

この定理を証明するため,幾つの準備sが必要である.
\begin{enumerate}
\item {\bf ランダムな$H$の構成}\\
確率的に証明するため,次のようにランダムな$H$を構成する:$H$のすべての要素を$0$にし,列ごとにランダムに$t$個の要素をスリップ($1 \to 0, 0\to 1$)する.ここで,$t$は密度パラメータである.
\item {\bf ノイズベクトルの確率$P(\bm{n})$}\\
あるノイズベクトル$\bm{n}$に対する確率は
\[
P(\bm{n}) := \prod_i^n (P(\bm{n}^{(i)}))
\]
この問題では,$P(\bm{n}^{(i)}=1)=p$,$P(\bm{n}^{(i)}=0)=1-p$ので,
\[
P(\bm{n}) := \prod_{\bm{n}^{(i)}=1} p \times \prod_{\bm{n}^{(i)}=0} 1-p
\]
\item {\bf 平均エントロピー(mean entropy)}\\
ノイズベクトル$\bm{n}$の平均エントロピーは
\[
H_{\bm{n}} := - \sum_{i}^n P(\bm{n}^{(i)}) \log_2 P(\bm{n}^{(i)}) = p\log_2(\frac{1}{p}) + (1-p)\log_2(\frac{1}{1-p})
\]
\item {\bf Typical set}\\
Typical setを次のように定める
\[
T = T_{n,\eta} = \Bigr\{ \bm{n} \in \{0,1\}^n : \Bigr|\frac{1}{n}\log_2\frac{1}{P(\bm{n})}-H_{\bm{n}}\Bigr| \le \eta \Bigr\}
\]
ただし,$\eta$はある小さい定数.$\sum_{\bm{n}\in T} P(\bm{n}) \le 1$ので,typical setの要素の数は
\[
|T| \le 2^{n(H_{\bm{n}}+\eta)}
\]
を満たす.
\end{enumerate}

大数の法則より,$n\to\infty,\bm{n}\in \{\bm{n}|\bm{n}\notin T \}=\emptyset$ので,$\bm{n}\in \{\bm{n}|\bm{n}\notin T \}$だけを考えばよい.

また,
\[
\delta(\text{論理式})=\begin{cases}
1& \text{論理式はtrue};\\
0& \text{論理式はfalse}
\end{cases}
\]
となる$\delta(\cdot)$を定義すると,
\[
P^*(H) = P_{H,\bm{n}_*}[\bm{n}_*\text{が}\bm{c}=G(\bm{n}_*,H)\text{のただ一つの最尤解である}]
\]
の確率の上界は次である.
\[
P^*(H) \le \sum_{\bm{n}\in T} P(\bm{n}) \sum_{\substack{\bm{n}'\in T,\\ \bm{n}' \neq \bm{n}}} \delta[H(\bm{n} - \bm{n}') \neq 0]
\]
また,$H$はランダムで構造されるので,$H$の平均をとると,
\[
\bar{P}^* \le \sum_{\substack{\bm{n}, \bm{n}'\in T, \\\bm{n}\neq \bm{n}'}} P(\bm{n}) \Bigr\{ \sum_{H} P(H) \delta[H(\bm{n} - \bm{n}') \neq 0] \Bigr\}
\]
\[
\
\]

\end{proof}

\end{document}