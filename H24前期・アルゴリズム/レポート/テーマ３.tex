\documentclass[a4paper,11pt]{jsarticle}
\usepackage{graphicx}
\usepackage{wrapfig}
\usepackage{amsmath,amssymb,amsthm}
\usepackage{amssymb}
\usepackage{ascmac}
\usepackage{subfigure}
\usepackage{bm}
\usepackage{setspace}
\usepackage{cases}%左かっこつけるときに必要だった
\usepackage{leftidx}%行列表示用?
\usepackage{fancyhdr}
\usepackage{graphicx}
\usepackage{float}
\usepackage{booktabs}
\usepackage{url}
\usepackage{bm}
\usepackage{verbatim}
\usepackage{calc}

\setlength{\headsep}{5mm}
\setlength{\oddsidemargin}{-0.5zw} %→にズラす
\setlength{\textheight}{37\baselineskip}
\addtolength{\textheight}{\topskip}
\setlength{\topmargin}{-10mm}
\setlength{\textwidth}{45zw} %文章の幅
\setlength{\textheight}{215mm}
\setlength{\parindent}{1zw}%箇条書きの一文字下げ
\pagestyle{fancy}

\newtheorem{theorem}{定理}
\newtheorem{prop}[theorem]{命題}
\newtheorem{lemma}[theorem]{補題}
\newtheorem{cor}[theorem]{系}
\newtheorem{example}[theorem]{例}
\newtheorem{definition}[theorem]{定義}
\newtheorem{rem}[theorem]{注意}
\newtheorem{guide}[theorem]{参考}
\renewcommand{\proofname}{証明}

\numberwithin{theorem}{section}  % 定理番号を「定理2.3」のように印刷
\numberwithin{equation}{section} % 式番号を「(3.5)」のように印刷
\newcommand{\argmax}{\mathop{\rm arg~max\,}\limits}
\newcommand{\argmin}{\mathop{\rm arg~min\,}\limits}
\newcommand{\st}{\mathop{\rm subject~to\,}\limits}
\newcommand{\sign}{\mathop{\rm sign\,}\limits}

\newcommand{\dom}{\mathop{\mathrm{\bf dom}}\nolimits}
\newcommand{\minimize}{\mathop{\mathrm{\rm minimize}}\limits}
\newcommand{\maximize}{\mathop{\mathrm{\rm maximize}}\limits}

\lhead{2012年度計算数理応用ーアルゴリズム・テーマ3レポート}
\chead{}
\rhead{12M42340}
\lfoot{チョウ シホウ}
\cfoot{\thepage}
\rfoot{6月11日}
\renewcommand{\footrulewidth}{0.4pt}
\title{2012年度計算数理応用ーアルゴリズム・テーマ3レポート}
\author{12M42340 チョウ シホウ}
\date{6月11日}

\begin{document}
\pagenumbering{roman}
%\tableofcontents
%\cleardoublepage
\pagenumbering{arabic}
\renewcommand{\thepart}{\arabic{part}}

%\thispagestyle{empty}
\begin{itembox}[l]{テーマ3・問2}
実際に{\bf AdaBoost}のプログラムを作成し,実験データ(例えばmushroom)を用いて,その性能などを実験し,その結果・解析・考察を述べよ.訓練データ(つまり事例集合)として$1000$個くらいを使い,残りのデータを使って得られた仮説の良さを評価してみるとよい.ベストな仮説は何か?訓練データを多くするとどうなるか?高速化の工夫と効果は?等々,いろいろと調べられると思う.
\end{itembox}

c言語でアルゴリズムの実装はpythonより難しいため,今回は{\bf pyclassic}のソースを参照し,pythonで{\bf Adaboost}のプログラムを作成した,参考先は\url{http://code.google.com/p/pyclassic/}.
アルゴリズムとしては,資料通りの伝統的なAdaboostを使う,たくさんの識別器も選ばれるが,ここでは決定株(Decision stump)という弱識別器だけを考える.

また,実験データはmushroomを用いる,今回のデータはshuffle済みなので,訓練集合をデータの前からの$m$個,テスト集合をデータの後からの$n$個,$m+n<\text{データの数}$とする.ここでは,$m\in\{500,1000,1500,2000\}$,$n\in\{500, 1000, 1500, 2000\}$,それぞれの$m,n$を組み合せ,訓練・予測を行う.閾値$\varepsilon=0.1$,反復回数$T=10$に設定する,実験結果は下記である.
\begin{table}[htdp]
\caption{データサイズ別の判別精度}
\begin{center}
\begin{tabular}{|c|c|c|c|c|c|}
\hline
訓練集合サイズ&テスト集合サイズ& 判別精度(\%) & 学習時間(s) & 予測時間(s) & 反復回数\\
\hline
$500$ & $500$ & $0.822000$ & 00.224870 & 00.000334 & $3$\\
$500$ & $1000$ & $0.837000$ & 00.191105 & 00.000409 & $3$\\
$500$ & $1500$ & $0.838667$ & 00.225717 & 00.000910 & $3$\\
$500$ & $2000$ & $0.842000$ & 00.195486 & 00.001162 & $3$\\
\hline
$1000$ & $500$ & $0.826000$ & 00.451063 & 00.000326 & $3$\\
$1000$ & $1000$ & $0.841000$ & 00.408634 & 00.000505 & $3$\\
$1000$ & $1500$ & $0.844667$ & 00.414288 & 00.000487 & $3$\\
$1000$ & $2000$ & $0.848500$ & 00.375509 & 00.000639 & $3$\\
\hline
$1500$ & $500$ & $0.826000$ & 00.574550 & 00.000239 & $3$\\
$1500$ & $1000$ & $0.841000$ & 00.537895 & 00.000383 & $3$\\
$1500$ & $1500$ & $0.844667$ & 00.562652 & 00.000431 & $3$\\
$1500$ & $2000$ & $0.848500$ & 00.548065 & 00.000555 & $3$\\
\hline
$2000$ & $500$ & $0.826000$ & 00.713411 & 00.000239 & $3$\\
$2000$ & $1000$ & $0.841000$ & 00.703578 & 00.000339 & $3$\\
$2000$ & $1500$ & $0.844667$ & 00.705330 & 00.000442 & $3$\\
$2000$ & $2000$ & $0.848500$ & 00.693032 & 00.000532 & $3$\\
\hline
\end{tabular}
\end{center}
\label{default}
\end{table}

訓練集合サイズだけを考察すると,$500$個の集合より$1000$個以上のほうが精度が高い,しかし,$1000$個,$1500$個,$2000$個の訓練集合の精度が等しい,オッカムのカミソリより,コンパクトな仮説のほうが望ましい,機械学習における,簡単なモデルのほうがoverfittingしにくいので,$1000$個の訓練集合のほうが効率よいと思う.一方,テスト集合サイズが大ければ大きいほど精度が高いに見られる.また,学習時間は訓練集合サイズとは正相関,学習時間と比べ,予測時間はかなり短いこと(早い過ぎで正確に測られない)も分かられる.反復回数を考えると,すべての学習は$3$反復まで終わるので,閾値に近づくのは早いである.閾値をそれぞれ$0.1,0.2,0.3,0.4$に設定し,実験をやり直したが,全く同じな結果が出るので,ここでは挙げない.

次は{\bf adaboost}の収束を考え,閾値を外し,$T=10,\varepsilon=0.1,m=1000,n=2000$の設定で各反復の計算結果を示す.ただし,$e=P_{\bm{\alpha},D_0} [f_*(\bm{\alpha})\neq f_t(\bm{\alpha})]$.

\begin{table}[htdp]
\caption{各反復の予測精度}
\begin{center}
\begin{tabular}{|c|c|c|c|}
\hline
反復($t$) & 誤判別率($e$) & 優位度($\gamma_t$)& 仮説の重み($\alpha_t$)\\
\hline
$1$ & $0.500000$ & $0.347000$ & $0.855631$\\
$2$ & $0.366000$ & $0.297096$ & $0.684122$\\
$3$ & $0.000000$ & $0.378471$ & $0.989016$\\
$4$ & $0.366000$ & $0.306668$ & $0.714251$\\
$5$ & $0.000000$ & $0.379926$ & $0.995865$\\
$6$ & $0.366000$ & $0.307153$ & $0.715808$\\
$7$ & $0.000000$ & $0.380001$ & $0.996219$\\
$8$ & $0.366000$ & $0.307178$ & $0.715889$\\
$9$ & $0.000000$ & $0.380005$ & $0.996237$\\
$10$ & $0.366000$ & $0.307179$ & $0.715893$\\
\hline
\end{tabular}
\end{center}
\label{default}
\end{table}
表通り,誤判別率は$0$反復でランダム誤判別率$0.5$から$2$反復ですぐ$0.000000$に減少するが,閾値を設定せずに計算し続くと,誤判別率が上がり,$0.366000$に戻ってしまい.優位度も同じく二つの値の間に繰り返す,むだな仮説も増加する.従って閾値の設定が必要だと思う.なお,資料よりパラメータを代入し,Adaboostのブースティング性を計算すると,反復回数$T\le 19.12$だが,実際にブースティングはかなり早いので,上界までは行かない.

高速化するため,次の方法を考えた.上記の結果より,学習時間の削減をメインに考える.
\begin{enumerate}
\item {\bf よりよい弱分類器を使う}\,\,\,\,
今回はDecision Stumpを実装したが,他の弱分類器を使ったほうが早いと思う.例えば,Hard marginのSVMならば,二次計画問題に定着できる(高次元に射影しなければいけないが);また,Lassoなどの$\ell_1$アルゴリズムを用い,疎性が高い解を求め,計算時間もメモリーの減少できる.一方,Adaboostで各反復で予測を行う時,重みしか更新されないので,もしオンライン学習のアルゴリズムを使い,更新された分だけをアップデートすれば,メモリーが削減できるし,早いスビードで収束することも可能になる.
\item {\bf 変数の計算}\,\,\,\, 例えば,$D_t$の更新する際,適当な$スデップサイズ$を追加すればもっと早く計算できると思う.
\item {\bf 各反復でのメモリー削減}\,\,\,\,
今回のプログラムは毎回予測を行う際すべての$\alpha_t$を使ったが,実際にはメモリーに記録し,更新分だけを追加すればよいと思う.


\end{enumerate}

\end{document}