\documentclass{jsarticle}
\topmargin=-3.0cm \oddsidemargin=0.1cm \evensidemargin=0.1cm
\textwidth=16 true cm \textheight=27 true cm
\setlength{\textheight}{250mm}

\usepackage{amsmath, amssymb}
\usepackage{fancybox,ascmac}

\begin{document}
\title{2012年度金融工学・第五回レポート}
\author{{\normalsize 社会理工学研究科 経営工学専攻 チョウ シホウ 学籍番号 12M42340}}
\date{}
\maketitle

\def \Pr{{\rm Pr}}


\baselineskip 0.6cm

\begin{itembox}[l]{5月16日・練習問題1}
株価:$\{ S_n \}_{n=0}^\infty$\\
収益率:$R_n = \frac{S_n - S_{n-1}}{S_{n-1}}, \,\,\, n \ge 1,\,\,\, \{x_n\}_{n=1}^\infty \text{ is i.i.d.}$\\
$E[x_n]=0,E[x_n^2]=1$とする.\\
今,$R_n = m + \sqrt{v} x_n, \,\,(m \in \mathbb{R},v>0)$と表されると仮定する(収益率のモデル化).このとき,期待収益率は$E[R_n]=m$,収益率の分散は$V[R_n]=E[(R_n-E[R_n])^2]=v$である.また,$S_n=S_{n-1}+S_{n-1}R_n=S_{n-1}(1+m+\sqrt{v}x_n)$,$\mathcal{F}_n = \sigma(x_1,\dots,x_n)$とおく,
\begin{enumerate}
\item $E[S_4|\mathcal{F}_2]=?$
\item $E[S_{n+3}|\mathcal{F}_n]=?$
\end{enumerate}
\end{itembox}
{\bf 解答:}
\begin{enumerate}
\item
$S_n=S_{n-1}(1+m+\sqrt{v}x_n)$ので,$S_n=S_0\prod_{i=1}^n (1+m+\sqrt{v}x_i)$.\\
条件付き期待値の性質:
\begin{enumerate}
\item $X$が$\mathcal{G}$-可測ならば,$E[X|\mathcal{G}] =X$
\item $X$と$\mathcal{G}$が独立ならば,$E[X|\mathcal{G}] = E[X]$
\end{enumerate}
より
\[
\begin{split}
E[S_4|\mathcal{F}_2] &=\,\,\, E[S_0\prod_{i=1}^4 (1+m+\sqrt{v}x_i) | \sigma(x_1,x_2)]\\
&=\,\,\, S_0E[\prod_{i=1}^2 (1+m+\sqrt{v}x_i) | \sigma(x_1,x_2)]E[\prod_{i=3}^4 (1+m+\sqrt{v}x_i) | \sigma(x_1,x_2)]\\
&=\,\,\, \Bigr( S_0\prod_{i=1}^2 (1+m+\sqrt{v}x_i) \Bigr) E[\prod_{i=3}^4(1+m+\sqrt{v}x_i)] \,\,\,\,\ \text{性質(a)(b)より} \\
&=\,\,\, S_2(1+m)^2
\end{split}
\]
\item 同様に,
\[
\begin{split}
E[S_{n+3}|\mathcal{F}_n] &=\,\,\, E[S_0\prod_{i=1}^{n+3} (1+m+\sqrt{v}x_i) | \sigma(x_1,\dots,x_n)]\\
&=\,\,\, S_0E[\prod_{i=1}^n (1+m+\sqrt{v}x_i) | \sigma(x_1,\dots,x_n)]E[\prod_{i=n+1}^{n+3} (1+m+\sqrt{v}x_i) | \sigma(x_1,\dots,x_n)]\\
&=\,\,\, \Bigr( S_0\prod_{i=1}^n (1+m+\sqrt{v}x_i) \Bigr) E[\prod_{i=n+1}^{n+3} (1+m+\sqrt{v}x_i)]\\
&=\,\,\, S_n(1+m)^3
\end{split}
\]
\end{enumerate}

\newpage
\begin{itembox}[l]{5月16日・練習問題2}
金利は$r$,$S_n^0=(1+r)^n$,株式の現在価値は$\tilde{S_n} = \frac{S_n}{S_n^0}$

$\{ \tilde{S_n}\}_{n=0}^\infty $がマルチンゲールになるための必要十分条件を求めよ.
%($\{ x_n\}_{n=0}^\infty \,\,\,\text{adapted},\,\,\,  E[x_n] < \infty, \forall n,\{x_n\}$)
\end{itembox}
{\bf 解答:}

マルチンゲールの定義より,$\{ \tilde{S_n}\}_{n=0}^\infty $が$\{\Omega,\mathcal{F},P\}$上のマルチンゲールになるための必要十分条件は
\begin{enumerate}
\item 任意の時刻$n$について$\tilde{S}_n$は$\mathcal{F}_n$可測確率変数である(Adapted).
\item 任意の時刻$n$について$\tilde{S}_n$は可積分である.
\[
\forall n > 0, \,\, E[\tilde{S}_n] < \infty
\]
\item 任意の時刻$n$について$t>0, E[\tilde{S}_{n+t}|\mathcal{F}_n]=\tilde{S}_n$.
\end{enumerate}
株式の現在価値$\tilde{S}_n$は
\[
\tilde{S}_n = \frac{S_0\prod_{i=1}^n (1+m+\sqrt{v}x_i)}{ (1+r)^n }
\]
$n$期までの情報を知り,$n+t$期以降の期待値は
\[
\begin{split}
E[\tilde{S}_{n+t}|\mathcal{F}_n] &=\,\,\, E\Bigr[ \frac{S_0\prod_{i=1}^{n+t} (1+m+\sqrt{v}x_i)}{ (1+r)^{n+t} } \Bigr|  \sigma(x_1,\dots,x_n)\Bigr]\\
&=\,\,\, E\Bigr[ \frac{S_0\prod_{i=1}^{n} (1+m+\sqrt{v}x_i)}{ (1+r)^{n+t} } \Bigr|  \sigma(x_1,\dots,x_n)\Bigr]E\Bigr[\prod_{i=n+1}^{n+t} (1+m+\sqrt{v}x_i) \Bigr|  \sigma(x_1,\dots,x_n)\Bigr]\\
&=\,\,\, E\Bigr[ \frac{S_0\prod_{i=1}^{n} (1+m+\sqrt{v}x_i)}{ (1+r)^{n+t} } \Bigr|  \sigma(x_1,\dots,x_n)\Bigr](1+m)^t\\
&=\,\,\, \tilde{S_n}\Bigr(\frac{1+m}{1+r}\Bigr)^t
\end{split}
\]
明らかに,$m=t$の時のみ,$E[\tilde{S}_{n+t}|\mathcal{F}_n]=\tilde{S}_n$が成り立つので,条件1,2と$m=t$三つの条件は$\{ \tilde{S_n}\}_{n=0}^\infty $がマルチンゲールになるための必要十分条件



\newpage
\begin{itembox}[l]{5月26日・練習問題1}
$\{w_t\}_{t \ge 0},\{\tilde{w}_t\}_{t \ge 0}$,1-dim,BM独立.\\
$\rho \in (-1,1), B_t := \rho w_t + \sqrt{1-\rho^2}\tilde{w}_t,(t \ge 0)$
\begin{enumerate}
\item $\{B_t\}_{t\ge 0}$がBMですることを確かめよ(特性関数を使うべし).
\item $B_t$と$w_t$の相関係数.
\end{enumerate}
\end{itembox}
{\bf 解答:}
\begin{enumerate}
\item
\begin{itemize}
\item
$B_0 = \rho w_0 + \sqrt{1-\rho^2} \tilde{w}_0 = 0$
\item
$w_t$と$\tilde{w}_t$は確率$1$(almost surely)で連続であるので,合成関数の連続性より,$B_t := \rho w_t + \sqrt{1-\rho^2}\tilde{w}_t$も連続.
\item
%{\bf 定常増分性}\\
$0 \le s < t$,$w_t-w_s$と$\tilde{w}_t - \tilde{w}_s$は定常増分性を持つので,平均$0$分散$t-s$の正規分布に従う,それぞれの特性関数は
\[
\begin{split}
E[\exp(i \alpha (w_t-w_s))] &=\,\,\, \exp(i\mu \alpha -\frac{1}{2}\sigma^2\alpha^2)\\
E[\exp(i \alpha (\tilde{w}_t-\tilde{w}_s))] &=\,\,\, \exp(i\tilde{\mu}\alpha-\frac{1}{2}\tilde{\sigma}^2\alpha^2)
\end{split}
\]
ただし,$\mu,\tilde{\mu},\sigma,\tilde{\sigma}$はそれぞれ$(w_t - w_s),(\tilde{w}_t - \tilde{w}_s)$の平均と分散である.次は,$B_t-B_s$の特性関数を求める.
\[
\begin{split}
E[\exp(i \alpha (B_t-B_s))] &= \,\,\, E[\exp(i \alpha (\rho w_t + \sqrt{1-\rho^2}\tilde{w}_t) - i \alpha (\rho w_s + \sqrt{1-\rho^2}\tilde{w}_s))] \\
&= \,\,\, E[\exp(i \alpha \rho ( w_t -  w_s)) \exp( i \alpha \sqrt{1-\rho^2}(\tilde{w}_t - \tilde{w}_s))]\\
&= \,\,\, E[\exp(i \alpha \rho ( w_t -  w_s))]E[ \exp( i \alpha \sqrt{1-\rho^2}(\tilde{w}_t - \tilde{w}_s))]\\
&=\,\,\, \exp \Bigr(i\mu \rho \alpha -\frac{1}{2}\sigma^2\alpha^2\rho^2 \Bigr) \exp \Bigr(i\tilde{\mu}\alpha\rho-\frac{1}{2}\tilde{\sigma}^2\alpha^2(1-\rho^2) \Bigr)\\
%&=\,\,\, E[\exp(it (w_t-w_s))^\rho]E[\exp(it (\tilde{w}_t-\tilde{w}_s))^{\sqrt{1-\rho^2}}]\\
%&=\,\,\, E[\exp(it (w_t-w_s))]^\rho E[\exp(it (\tilde{w}_t-\tilde{w}_s))]^{\sqrt{1-\rho^2}}\\
%&=\,\,\, \exp \Bigr(i\mu t-\frac{1}{2}\sigma^2t^2 \Bigr)^\rho \exp \Bigr(i\tilde{\mu}t-\frac{1}{2}\tilde{\sigma}^2t^2 \Bigr)^{\sqrt{1-\rho^2}}\\
%&=\,\,\, \exp \Bigr( i(\rho \mu + \sqrt{1-\rho^2}\tilde{\mu} )t - \frac{1}{2}( \rho \sigma^2 + \sqrt{1-\rho^2}\tilde{\sigma}^2)t^2 \Bigr)
%&=\,\,\, \exp \Bigr(-\frac{1}{2}t^2 \Bigr)^\rho \exp \Bigr(-\frac{1}{2}t^2 \Bigr)^{\sqrt{1-\rho^2}}\\
%&=\,\,\, \exp \Bigr(-\frac{1}{2}(\rho + {\sqrt{1-\rho^2}})t^2 \Bigr)
&=\,\,\, \exp \Bigr(i(\mu\rho + \tilde{\mu}\sqrt{1-\rho^2})\alpha -\frac{1}{2}(\rho^2 \sigma^2 + (1-\rho^2)\tilde{\sigma}^2)\alpha^2 \Bigr)
\end{split}
\]
従って,$B_t-B_{t-1}$は$N\Bigr(\rho \mu + \sqrt{1-\rho^2}\tilde{\mu} ,\rho^2 \sigma^2 + (1-\rho^2)\tilde{\sigma}^2 \Bigr)=N(0, t-s)$の正規分布に従い,{\bf 定常増分性}を持つ.
\item
%{\bf 独立増分性}\\
定常増分性より,$w_t-w_s$と$\tilde{w}_t - \tilde{w}_s$の平均も$0$である.すべての$0<n<m,0 \le s < t$に対して
\[
\begin{split}
\text{cov}(w_{t_m} - w_{s_m }, w_{t_n} - w_{s_n})=&E[(w_{t_m} - w_{s_m })(w_{t_n} - w_{s_n})]=0\\
\text{cov}(\tilde{w}_{t_m} - \tilde{w}_{s_m }, \tilde{w}_{t_n} - \tilde{w}_{s_n})=&E[(\tilde{w}_{t_m} - \tilde{w}_{s_m })(\tilde{w}_{t_n} - \tilde{w}_{s_n})]=0
\end{split}
\]
が成り立つ,また,$B_t-B_s$の平均は$0$であるので,
\[
\begin{split}
&\,\,\,  \text{cov}(B_{t_m} - B_{s_m }, B_{t_n} - B_{s_n})\\
=&\,\,\, E[(B_{t_m} - B_{s_m })(B_{t_n} - B_{s_n})]\\
=&\,\,\, E \Bigr[ \Bigr( \rho (w_{t_m} - w_{s_m}) + \sqrt{1-\rho^2}(\tilde{w}_{t_m} - \tilde{w}_{s_m}) \Bigr)\Bigr( \rho (w_{t_n} - w_{s_n}) + \sqrt{1-\rho^2}(\tilde{w}_{t_n} - \tilde{w}_{s_n}) \Bigr)  \Bigr]\\
=&\,\,\, \rho^2 E \Bigr[   (w_{t_m} - w_{s_m})(w_{t_n} - w_{s_n}) \Bigr] + (1-\rho^2)E \Bigr[ (\tilde{w}_{t_m} - \tilde{w}_{s_m})(\tilde{w}_{t_n} - \tilde{w}_{s_n})\Bigr]  + \\
&\,\,\,\,\,\,\,\,\,\rho\sqrt{1-\rho^2}\Bigr( E \Bigr[(w_{t_n} - w_{s_n})(\tilde{w}_{t_m} - \tilde{w}_{s_m})\Bigr] +  E \Bigr [(w_{t_m} - w_{s_m})(\tilde{w}_{t_n} - \tilde{w}_{s_n}) \Bigr] \Bigr)\\
=&\,\,\, \rho\sqrt{1-\rho^2}\Bigr( E \Bigr[(w_{t_n} - w_{s_n})(\tilde{w}_{t_m} - \tilde{w}_{s_m})\Bigr] +  E \Bigr [(w_{t_m} - w_{s_m})(\tilde{w}_{t_n} - \tilde{w}_{s_n}) \Bigr] \Bigr)\\
=&\,\,\, 0
\end{split}
\]
従って,{\bf 独立増分性}を持つ.\\
以上の四つの条件を満たすので,$\{B_t\}_{t\ge0}$はBMである.
\end{itemize}
\item
BMの定常増分性より
\[ B_t = B_t - B_0 = N(0, t) \]
\[ w_t = w_t - w_0 = N(0,t) \]
\[ \tilde{w}_t = \tilde{w}_t - \tilde{w}_0 = N(0,t) \]
$B_t$と$w_t$の相関係数$\theta$は
\[
\begin{split}
\theta &=\,\,  \frac{ E[B_t w_t] - E[B_t]E[w_t] }{ \sqrt{ (E[B_t^2] - E[B_t]^2) (E[w_t^2] - E[w_t]^2) } }\\
&=\,\,  \frac{ E[B_t w_t] }{ \sqrt{ t^2 }}\\
&=\,\,  \frac{ E[ (\rho w_t + \sqrt{1-\rho^2}\tilde{w}_t)w_t ] }{ t}\\
&=\,\,  \frac{ \rho E[w_t^2] + \sqrt{1-\rho^2}E[\tilde{w}_t w_t] }{ t}\\
&=\,\,  \frac{ \rho E[w_t^2] }{ t}\\
&=\,\,  \rho
\end{split}
\]

\end{enumerate}

\end{document}
