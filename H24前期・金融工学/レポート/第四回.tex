\documentclass{jsarticle}
\topmargin=-3.0cm \oddsidemargin=0.1cm \evensidemargin=0.1cm
\textwidth=16 true cm \textheight=27 true cm
\setlength{\textheight}{250mm}

\usepackage{amsmath, amssymb}
\usepackage{fancybox,ascmac}

\begin{document}
\title{2012年度金融工学・第四回レポート}
\author{{\normalsize 社会理工学研究科 経営工学専攻 チョウ シホウ 学籍番号 12M42340}}
\date{}
\maketitle

\def \Pr{{\rm Pr}}


\baselineskip 0.6cm

\begin{itembox}[l]{問題6-3-1}
Fatou's lemmaに於いて,等号が成立しない様な関数列の例を作れ.
\end{itembox}
{\bf 解答:}

Fatou's lemmaは非負可測関数しか適用できないことを基に次の例を構成する\\
$(\mathcal{S},\mathcal{B},\mu)$は$[0,+\infty)$上の測度空間,$\mathcal{B}$はボレルσ-algebra,$\mu$はレベーク測度、すべての自然数$n$に対して$f_n$を次に定義する
\[
f_n(x) = \begin{cases}
\frac{1}{n} & \text{\,\,\,\,\,\, for $x \in [n,2n]$}, \\
0 & \text{\,\,\,\,\,\,otherwise}
\end{cases}
\]

関数$f_n$は$\mathcal{S}$から$0$に一様収束し、すべての$x \le 0,n > 0$に対しては$f_n(x)=0$だが、すべての$f_n$の積分が$1$であるので、
\[
0 = \mu(\liminf_n f_n) = \int_{\mathcal{S}} \liminf_{n \to \infty} f_n d\mu < \liminf_{n \to \infty} \int_{\mathcal{S}} f_n d\mu =  \liminf_n \mu(f_n)  = 1
\]

等式が成立しない。
\begin{itembox}[l]{問題6-3-2}
\begin{enumerate}
\item Reverse Fatou's lemmaを証明し,等号の成立しない例を作れ.
\item $\mu(g) < + \infty$ の条件が満たされない場合に,上の不等式が成立しなくなる様な例を作れ.
\end{enumerate}
\end{itembox}
{\bf 解答:}
\begin{enumerate}
\item
$\exists g, \forall n \in \mathbb{N}, \mu(g) < + \infty, f_n \le g$ので,関数列$h_n=(g-f_n)$とおき,$\forall{h_n}_{n\in \mathbb{N}} \subset (m\mathcal{B})^+$ので,Fatou's lemmaを適用できる,
\begin{equation*}
\mu(\liminf_n (g-f_n)) \le \liminf_n \mu(g-f_n)
\end{equation*}
また,$\liminf_n (-f_n) = -\limsup_n f_n$より,不等式の左辺は,
\begin{equation*}
\begin{split}
\mu(\liminf_n (g-f_n)) &=\,\,\, \mu(g + \liminf_n (- f_n)) \\
&=\,\,\, \mu(g)  + \mu(\liminf_n (-f_n))\\
&=\,\,\, \mu(g) + \mu(-\limsup_n f_n)\\
&=\,\,\, \mu(g) - \mu(\limsup_n f_n)
\end{split}
\end{equation*}
に変形できる.一方,不等式の右辺は
\begin{equation*}
\begin{split}
\liminf_n \mu(g-f_n) &=\,\,\, \liminf_n (\mu(g)-\mu(f_n))\\
&=\,\,\, \mu(g) + \liminf_n(-\mu(f_n))\\
&=\,\,\, \mu(g) - \limsup_n \mu(f_n)\\
\end{split}
\end{equation*}
になるので.両辺$\mu(g)$を引く,不等式の向きを逆にすれば,
\begin{equation*}
 \limsup_n \mu(f_n) \le  \mu(\limsup_n f_n)
\end{equation*}
次は,反例一個を挙げる:
$\mathcal{S}$を$\mathbb{R}$上の測度空間に定義し、関数列$f_n\in \mathbb{N}$を
\[
f_n(x) = \begin{cases}
-\frac{1}{n} & \text{\,\,\,\,\,\, for $x \in [n,2n]$}, \\
0 & \text{\,\,\,\,\,\,otherwise}
\end{cases}
\]
とおくと、可積分関数$g(x)=0$が存在し、$\forall n \in \mathbb{N}, f_n \le g$ので,非負列$(g-f_n)$が上記の例により等式が成り立たないので、逆向きのFatou's lemmaの等式も成立しない。
\[
0 = \limsup_{n \to \infty} \mu(f_n) = \limsup_{n \to \infty} \int_{\mathcal{S}} f_n d\mu < \int_{\mathcal{S}} \limsup_{n \to \infty} f_n d\mu = \mu( \limsup_{n \to \infty} f_n) = 1
\]

\item 前の証明を考え,$\mu(g)>+\infty$が成り立たない場合,両方$\mu(g)$を引くと不等式が必ず満たさないので,上の不等式が成り立たない.

下記の反例を考え,
\[
f_n(x) = x^2
\]
$\forall g, \mu(g) = + \infty$ので,不等式が成り立たない

\end{enumerate}

\begin{itembox}[l]{問題6-4-1}
命題6.4.1(線形性)を証明せよ
\end{itembox}
{\bf 解答:}
$f,g$を下から近似する次の非負単関数の列$f_n,g_n$を考え,
\begin{equation*}
\begin{split}
\{f_n\}_{n \in \mathbb{N}}  = & \sum_{i=1}^n a_i 1_{A_i}(x) \to f \\
\{g_m\}_{m \in \mathbb{N}} = & \sum_{j=1}^m b_j 1_{B_j}(x) \to g
\end{split}
\end{equation*}
すると
\begin{equation*}
\begin{split}
&\,\,\, \mu(\alpha f_n + \beta g_n) \\
=&\,\,\, \mu \Bigr( \alpha \sum_{i=1}^n a_i 1_{A_i}(x)+ \beta \sum_{j=1}^m b_j 1_{B_j}(x) \Bigr)\\
=&\,\,\, \mu \Bigr( \alpha\sum_{i=1}^n a_i 1_{A_i}(x) \Bigr) + \mu \Bigr( \beta  \sum_{j=1}^m b_j 1_{B_j}(x)\Bigr),\,\,\,\,\text{非負単関数の加法性より}\\
=&\,\,\, \alpha \mu( \sum_{i=1}^n a_i 1_{A_i}(x)) +  \beta \mu(\sum_{j=1}^m b_j 1_{B_j}(x)),\,\,\,\,\text{非負単関数の正斉次性より}\\
=&\,\,\, \alpha \mu(f_n) + \beta \mu(g_m) \\
\end{split}
\end{equation*}
単調収束定理より,
\begin{equation*}
\mu(f_n) \to \mu(f),\,\, \mu(g_m) \to \mu(g),\,\,\mu(\alpha f_n + \beta g_m) \to \mu(\alpha f + \beta g)
\end{equation*}

\begin{itembox}[l]{問題6-5-1}
$ f  \in \mathcal{L}'(\mathcal{S},\mathcal{B},\mu),g_n = f1_{[-n,n]},h_n = \min (f,n)$と定める時,
\begin{equation*}
\mu(|f-g_n|) \to 0
\end{equation*}
\begin{equation*}
\mu(|f-h_n|) \to 0
\end{equation*}
の成立を示せ.
\end{itembox}
{\bf 解答:}
\begin{itemize}
\item $n \to \infty$の時,$g_n$は$f$に近似しているので,$g_n \to f$である.明らかに,$| f_n | \le | f |$,また,$\forall f \in m \mathcal{B} \Rightarrow \mu(|f|) < +\infty $ので,優収束定理より,
\begin{equation}
\mu(|f_n - f|) \to 0
\end{equation}
が成立である.
\item $h_n = \min(f,n)$ので,$n \to \infty$の時,$h_n$は$f$に近似しているので,$h_n \to f$である.なお,$\exists f'=|f|, |f_n| \ge f',\mu(f') < + \infty$にので,優収束定理より,
\begin{equation}
\mu(|h_n - f|) \to 0
\end{equation}
が成立である.
\end{itemize}

\begin{itembox}[l]{問題7-1-1}
Markovの不等式を証明せよ.
\end{itembox}
{\bf 解答:}
$I_A$特性確率関数を次のように定義する,
\begin{equation*}
I_A = 
\begin{cases}
1,\,\,\,\text{Aが起きる}\\
0,\,\,\,\text{Aが起きらない}
\end{cases}
\end{equation*}
する.明らかに,$I_{X \ge c} \le X$が成立,$g > 0$,ゆえに
\begin{equation*}
\begin{split}
g(c)P(\{ w | X(w) \ge c \}) &= \,\,\, g(c)E(I_{X(w)\ge c};\{ w| X(w) \ge c \})\\
&\le \,\,\, g(c)E(X;\{ w| X(w) \ge c \})\\
&= \,\,\, E(g\circ X;\{ w| X(w) \ge c \})
\end{split}
\end{equation*}
が成り立つ.また,$\{g\circ X|\{ w| X(w) \ge c \}\} \subseteq \{ g\circ X\}$ので,期待値の線形性による
\begin{equation}
E(g\circ X;\{ w| X(w) \ge c \}) \le E(g\circ X) \\
\end{equation}

\begin{itembox}[l]{添付問題}
次の不等式をMarkovの不等式から証明せよ.$\forall \theta >0, \forall c \in \mathbb{R}, \forall Y \in m\mathcal{F}$
\begin{equation*}
P(\{w|Y(w)>c\}) \le e^{-c\theta}E[e^{\theta Y}]
\end{equation*}
\end{itembox}
{\bf 解答:}
$g(c)=e^{c\theta}$とおくと,$(g \circ Y)(c) = e^{\theta Y(c)}$である.また,Markovの不等式により,
\begin{equation}
\begin{split}
g(c)P(\{w|Y(w) > c\}) &\,\,\,= e^{c\theta}P(\{w|Y(w) > c\})\\
&\,\,\, \le E(g \circ Y; \{ w|Y(w) > c \})\\
&\,\,\, \le E(g \circ Y) \\
&\,\,\, = E(e^{\theta Y})
\end{split}
\end{equation}
が成り立つ,$c>0,\theta>0 \Rightarrow e^{c\theta}>0$ので
\begin{equation}
P(\{w|Y(w) > c\})  \le e^{-c\theta}E(e^{\theta Y})
\end{equation}
\end{document}
