\documentclass{jsarticle}
\topmargin=-3.0cm \oddsidemargin=0.1cm \evensidemargin=0.1cm
\textwidth=16 true cm \textheight=27 true cm
\setlength{\textheight}{250mm}

\usepackage{amsmath, amssymb}
\usepackage{fancybox,ascmac}

\begin{document}
\title{2012年度金融工学・第六回レポート}
\author{{\normalsize 社会理工学研究科 経営工学専攻 チョウ シホウ 学籍番号 12M42340}}
\date{}
\maketitle

\def \Pr{{\rm Pr}}


\baselineskip 0.6cm

\begin{itembox}[l]{6月1日・練習問題1}
$\{w_t\}$は1-dimのBM.
\begin{enumerate}
\item $X_t := e^{\frac{t}{2}}\cos w_t$がマルチンゲールであることを示せ.(伊藤の公式を使う).
\item $x > 0, x_t = (x^{\frac{1}{3}} + \frac{1}{3} w_t)^3$が満たすSDEを導け.
\end{enumerate}
\end{itembox}

\begin{enumerate}
\item
各項の微分を取る
\begin{eqnarray*}
\frac{dX_t}{dt} &=& \frac{1}{2}e^{\frac{t}{2}}\cos w_t - e^{\frac{t}{2}}\sin w_t dw_t = \frac{1}{2}e^{\frac{t}{2}}\cos w_t \\
&& \text{\,\,\,\,\,\,\,\,\,\,\,\,\,\,\,\,\,\,\,\,(伊藤のルールより}dw_t dt = 0\text{)} \\
\frac{dX_t}{dw_t} &=& -e^{\frac{t}{2}}\sin w_t \\
\frac{d^2 X_t}{dw_t^2} &=& \frac{d^2 X_t}{dt} = -e^{\frac{t}{2}}\cos w_t \\
\end{eqnarray*}
すると
\[
dX_t = \frac{1}{2}e^{\frac{t}{2}}\cos w_t dt -e^{\frac{t}{2}}\sin w_t dw_t - \frac{1}{2}e^{\frac{t}{2}}\cos w_t dt = -e^{\frac{t}{2}}\sin w_t dw_t  \,\,\,\,\,\,\,\,\,\,\,\,\text{(伊藤の公式より)} 
\]
\[
E[dX_t|\mathcal{F}_t] = -e^{\frac{t}{2}}\sin w_t E[ dw_t | \mathcal{F}_t] =0
\]
従って,$X_t := e^{\frac{t}{2}}\cos w_t$がマルチンゲールである.

\item
\begin{eqnarray*}
\frac{dx_t}{dx} &=& 3(x^{\frac{1}{3}} + \frac{1}{3} w_t)^2 \frac{1}{3} x^{-\frac{2}{3}} = (x^{\frac{1}{3}} + \frac{1}{3} w_t)^2 x^{-\frac{2}{3}} \\
\frac{dx_t}{dw_t} &=& 3(x^{\frac{1}{3}} + \frac{1}{3} w_t)^2 \frac{1}{3} = (x^{\frac{1}{3}} + \frac{1}{3} w_t)^2\\
\frac{d^2x_t}{dw_t^2} &=& 2(x^{\frac{1}{3}} + \frac{1}{3} w_t) \frac{1}{3} =  \frac{2}{3}(x^{\frac{1}{3}} + \frac{1}{3} w_t)\\
\end{eqnarray*}
伊藤の公式より
\[
dx_t = x^{-\frac{2}{3}} (x^{\frac{1}{3}} + \frac{1}{3} w_t)^2 dx + (x^{\frac{1}{3}} + \frac{1}{3} w_t)^2 dw_t + \frac{1}{3}(x^{\frac{1}{3}} + \frac{1}{3} w_t) dw_t^2
\]

\end{enumerate}
\newpage

\begin{itembox}[l]{6月8日・練習問題1}

\begin{equation}
\text{SDE}
\begin{cases}
dX_t = \frac{1}{2} \sigma(X_t)\sigma'(X_t)dt + \sigma(X_t)dw_t \\
X_0 = x_0
\end{cases}を考え
\end{equation}
\begin{enumerate}
\item $f(x) = \int_{x_0}^x \frac{dy}{\sigma(y)}$として,$(f^{-1})',(f^{-1})''$を計算せよ
\item (1)の結果を利用して上のSDEを解け.
\end{enumerate}
\end{itembox}

\begin{enumerate}
\item 
\[dX_t = b(t,X_t)dt + \sigma(t,X_t)dw_t,\,\,\, \sigma = 0 \]
\[
dX_t = b(t,X_t)dt \Rightarrow \frac{dX_t}{dt} = b(t,X_t)
\]
\[
dX_t = \frac{1}{2} \sigma(X_t)\sigma'(X_t)dt + \sigma(X_t)dw_t
\]
\[
f(x) = \int_{x_0}^x \frac{dy}{\sigma(y)}
\]とおくと
\begin{eqnarray*}
(f^{-1})'(y) &=& \frac{1}{f'(f^{-1}(y))} = \sigma(f^{-1}(y)) \\
(f^{-1})''(y) &=& \sigma'(f^{-1}(y))(f^{-1})'(y) \\
&=& \sigma'(f^{-1}(y))\sigma(f^{-1}(y))
\end{eqnarray*}
\item
伊藤の公式より,$X_t=f^{-1}(w_t + f(x_0))$ので,
\begin{eqnarray*}
dX_t &=& df^{-1}(w_t + f(x_0))\\
&=& (f^{-1})'(w_t + f(x_0))dw_t + \frac{1}{2}(f^{-1})''(w_t + f(x_0))dt\\
&=& \sigma(f^{-1}(w_t + f(x_0)))dw_t + \frac{1}{2}\sigma'(f^{-1}(w_t + f(x_0)))\sigma(f^{-1}(w_t + f(x_0)))dt\\
&=& \sigma(X_t)dw_t + \frac{1}{2}\sigma'(X_t)\sigma(X_t)d_t
\end{eqnarray*}
従って
\[
X_0 = f^{-1}(w_0 + f(x_0)) = x_0 
\]
SDEの解である.

\end{enumerate}
\newpage
\begin{itembox}[l]{6月8日・練習問題2}
\begin{equation}
\text{SDE}
\begin{cases}
dX_t = dt + 2\sqrt{X_t} dw_t \\
X_0 = x_0
\end{cases}
\end{equation}
を解け
\end{itembox}

$\sigma(x)=2\sqrt{x}$とおく,
\[
dX_t = dt + 2\sqrt{X_t} dw_t,\,\,\, \sigma'(x)=\frac{1}{\sqrt{x}},\,\,\, \sigma\sigma'(x)=2
\]
によると
\[
dX_t = \frac{1}{2}\sigma(X_t)\sigma'(X_t)dt + \sigma(X_t)dw_t
\]
伊藤の公式により,
\begin{eqnarray*}
f(x) &=& \int_{x_0}^x \frac{dy}{2\sqrt{y}} = \int_{x_0}^x (\sqrt{y})'dy = \sqrt{x} - \sqrt{x_0} \\
f^{-1}(y) &=& (y + \sqrt{x_0})^2,\,\,\,\,\,\,\,\,\, f(x_0)=0
\end{eqnarray*}
従って,
\[
X_t = f^{-1}(w_t + f(x_0)) = (w_t + \sqrt{x_0})^2
\]


\end{document}
