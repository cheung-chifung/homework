\documentclass[a4paper,11pt]{jsarticle}
\usepackage{graphicx}
\usepackage{wrapfig}
\usepackage{amsmath,amssymb,amsthm}
\usepackage{amssymb}
\usepackage{ascmac}
\usepackage{subfigure}
\usepackage{bm}
\usepackage{setspace}
\usepackage{cases}%左かっこつけるときに必要だった
\usepackage{leftidx}%行列表示用?
\usepackage{fancyhdr}
\usepackage{graphicx}
\usepackage{float}
\usepackage{booktabs}
\usepackage{url}
\usepackage{bm}
\usepackage{verbatim}
\usepackage{calc}
\usepackage{algorithm}
\usepackage{algorithmic}

\setlength{\headsep}{5mm}
\setlength{\oddsidemargin}{-0.5zw} %→にズラす
\setlength{\textheight}{37\baselineskip}
\addtolength{\textheight}{\topskip}
\setlength{\topmargin}{-10mm}
\setlength{\textwidth}{45zw} %文章の幅
\setlength{\textheight}{215mm}
\setlength{\parindent}{1zw}%箇条書きの一文字下げ
\pagestyle{fancy}

\newtheorem{theorem}{定理}
\newtheorem{prop}[theorem]{命題}
\newtheorem{lemma}[theorem]{補題}
\newtheorem{cor}[theorem]{系}
\newtheorem{example}[theorem]{例}
\newtheorem{definition}[theorem]{定義}
\newtheorem{rem}[theorem]{注意}
\newtheorem{guide}[theorem]{参考}
\renewcommand{\proofname}{証明}

\numberwithin{theorem}{section}  % 定理番号を「定理2.3」のように印刷
\numberwithin{equation}{section} % 式番号を「(3.5)」のように印刷
\newcommand{\argmax}{\mathop{\rm arg~max\,}\limits}
\newcommand{\argmin}{\mathop{\rm arg~min\,}\limits}
\newcommand{\st}{\mathop{\rm subject~to\,}\limits}
\newcommand{\sign}{\mathop{\rm sign\,}\limits}

\newcommand{\dom}{\mathop{\mathrm{\bf dom}}\nolimits}
\newcommand{\minimize}{\mathop{\mathrm{\rm minimize}}\limits}
\newcommand{\maximize}{\mathop{\mathrm{\rm maximize}}\limits}

\newcommand{\prox}{\mathop{\mathrm{\rm Prox}}\limits}

\lhead{H24前期・バイオインフォマティクス・最終レポート}
\chead{}
\rhead{}
\lfoot{12M42340 チョウ シホウ}
\cfoot{\thepage}
\rfoot{}
\renewcommand{\footrulewidth}{0.4pt}


\begin{document}
\pagenumbering{roman}
%\tableofcontents
%\cleardoublepage
\pagenumbering{arabic}
\renewcommand{\thepart}{\arabic{part}}

%\thispagestyle{empty}

\begin{itembox}[l]{B群}
配列のペアワイズなローカルアライメント.対象はDNAでもアミノ酸配列でも可.
\end{itembox}
次の仕様でLocal alignment algorithmを実装した.
\begin{itemize}
\item MATCH\_AWARD = 2,DISMATCH\_PENALTY = -1,GAP\_PENALTY = -1,はそれぞれmatch,unlatch,gapの採点
\item m,nはそれぞれ文字列sa,sbの長さ,sa[i],sb[j]はそれぞれsa,sbの要素
\item scoreは採点行列$H$,pointerは最適方向行列
\end{itemize}

\section{アルゴリズム}
\begin{enumerate}
\item 初期化:\\
$H(i,0)=0, 0 \le i \le m;\,\,\,\, H(0,j)=0, 0 \le j \le n$\\
$\text{max score}=0$
\item 採点行列$H$と最適方向行列$D$を計算(各$(i,j),1\le i\le m, 1\le j \le n $):\\
d=match/unmatchの時$H(i-1,j-1)$を加点・ペナルティした点数;\\
l=delectionの時の$H(i-1,j)+\text{GAP\_PENALTY}$;\\
u=insertionの時の$H(i,j-1)+\text{GAP\_PENALTY}$\\
$H(i,j)=$MAX(d,l,u,0)\\
$D(i,j)=$高い点数の方向,優先順位は対角,上(insertion),左(delection),点数が0ならばストップ\\
max scoreを更新する,max scoreの位置を記録する.
\item trace back\\
max scoreの位置から最適方向行列$D$で記録した方向を用いて文字列を生成する,
\end{enumerate}

\section{テスト例}
テスト用文字列はそれぞれwikipediaと配布資料\#3 Sequene Alignmentの例である.
\begin{itemize}
\item
入力:文字列sa,sb
\item
出力:採点行列$H$,最適アライメントの座標表示,最高点数,最適アライメントの文字列
\end{itemize}
\newpage
\begin{enumerate}
\item
\begin{verbatim}
# input
sa="ACACACTA"
sb="AGCACACA"
# output
[[  0.   0.   0.   0.   0.   0.   0.   0.   0.]
 [  0.   2.   1.   2.   1.   2.   1.   0.   2.]
 [  0.   1.   1.   1.   1.   1.   1.   0.   1.]
 [  0.   0.   3.   2.   3.   2.   3.   2.   1.]
 [  0.   2.   2.   5.   4.   5.   4.   3.   4.]
 [  0.   1.   4.   4.   7.   6.   7.   6.   5.]
 [  0.   2.   3.   6.   6.   9.   8.   7.   8.]
 [  0.   1.   4.   5.   8.   8.  11.  10.   9.]
 [  0.   2.   3.   6.   7.  10.  10.  10.  12.]]
cell(8,8):diagonal
cell(7,7):left
cell(6,7):diagonal
cell(5,6):diagonal
cell(4,5):diagonal
cell(3,4):diagonal
cell(2,3):diagonal
cell(1,2):up
cell(1,1):diagonal
max score:12
result:
A-CACACTA
AGCACAC-A
\end{verbatim}
\item
\begin{verbatim}
# input
sa="GCTCGTTG"
sb="AACCGTAA"
# output
[[ 0.  0.  0.  0.  0.  0.  0.  0.  0.]
 [ 0.  0.  0.  0.  0.  0.  0.  0.  0.]
 [ 0.  0.  0.  0.  0.  0.  0.  0.  0.]
 [ 0.  0.  2.  1.  2.  1.  0.  0.  0.]
 [ 0.  0.  2.  1.  3.  2.  1.  0.  0.]
 [ 0.  2.  1.  1.  2.  5.  4.  3.  2.]
 [ 0.  1.  1.  3.  2.  4.  7.  6.  5.]
 [ 0.  0.  0.  2.  2.  3.  6.  6.  5.]
 [ 0.  0.  0.  1.  1.  2.  5.  5.  5.]]
cell(6,6):diagonal
cell(5,5):diagonal
cell(4,4):diagonal
cell(3,3):left
cell(2,3):diagonal
max score:7
result:
CTCGT
C-CGT
\end{verbatim}
\end{enumerate}
\end{document}