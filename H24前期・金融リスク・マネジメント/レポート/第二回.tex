\documentclass[a4paper,11pt]{jsarticle}
\usepackage{graphicx}
\usepackage{wrapfig}
\usepackage{amsmath,amssymb,amsthm}
\usepackage{amssymb}
\usepackage{ascmac}
\usepackage{subfigure}
\usepackage{bm}
\usepackage{setspace}
\usepackage{cases}%左かっこつけるときに必要だった
\usepackage{leftidx}%行列表示用?
\usepackage{fancyhdr}
\usepackage{graphicx}
\usepackage{float}
\usepackage{booktabs}
\usepackage{url}
\usepackage{bm}
\usepackage{verbatim}
\usepackage{calc} 

\setlength{\headsep}{5mm}
\setlength{\oddsidemargin}{-0.5zw} %→にズラす
\setlength{\textheight}{37\baselineskip}
\addtolength{\textheight}{\topskip}
\setlength{\topmargin}{-10mm}
\setlength{\textwidth}{45zw} %文章の幅
\setlength{\textheight}{215mm}
\setlength{\parindent}{1zw}%箇条書きの一文字下げ
\pagestyle{fancy}

\newtheorem{theorem}{定理}
\newtheorem{prop}[theorem]{命題}
\newtheorem{lemma}[theorem]{補題}
\newtheorem{cor}[theorem]{系}
\newtheorem{example}[theorem]{例}
\newtheorem{definition}[theorem]{定義}
\newtheorem{rem}[theorem]{注意}
\newtheorem{guide}[theorem]{参考}
\renewcommand{\proofname}{証明}

\numberwithin{theorem}{section}  % 定理番号を「定理2.3」のように印刷
\numberwithin{equation}{section} % 式番号を「(3.5)」のように印刷
\newcommand{\argmax}{\mathop{\rm arg~max\,}\limits}
\newcommand{\argmin}{\mathop{\rm arg~min\,}\limits}
\newcommand{\st}{\mathop{\rm subject~to\,}\limits}
\newcommand{\sign}{\mathop{\rm sign\,}\limits}

\newcommand{\dom}{\mathop{\mathrm{\bf dom}}\nolimits}
\newcommand{\minimize}{\mathop{\mathrm{\rm minimize}}\limits}
\newcommand{\maximize}{\mathop{\mathrm{\rm maximize}}\limits}

\newcommand{\prox}{\mathop{\mathrm{\rm Prox}}\limits}

\title{2012年度金融リスクマネジメント・第二回レポート}
\author{12M42340 チョウ シホウ}
\date{}

\begin{document}
\pagenumbering{roman}
%\tableofcontents
%\cleardoublepage
\pagenumbering{arabic}
\renewcommand{\thepart}{\arabic{part}}

\maketitle
\begin{itembox}[l]{問題1}
\[
\alpha (t,T) = v_T(t,T)v(t,T)-m_T(t,T)\,\,\,\,\,\,
\sigma(t,T) + v_T(t,T) = 0
\]を証明する
\end{itembox}
下記の記号を定める
\[
\begin{cases}
B(t)=(B_1(t),\dots,B_N(t))^\intercal \\
v(t,T)=(v_1(t,T),\dots,v_N(t,T))^\intercal \\
b(t)=(b_1(t),\dots,b_N(t))^\intercal \\
\sigma(t,T) = (\sigma_1(t,T),\dots,\sigma_N(t,T))^\intercal\\
\end{cases}
\]
よって,zero-coupon bondの価格
\[
P(t,T) = P(0,T) + \int_0^t P(s,T)m(s,T)ds + \int_0^t P(s,T)v(s,T)dB(s)
\]にItoの公式を適用すると,
\[
d P(t,T) = P(s,T)m(s,T)ds + P(s,T)v(s,T)dB(s)
\]
一方,$F(P(t,T))=\log P(t,T)$に定め,$F(P(t,T))$に対してItoの補題を適用する
\begin{eqnarray*}
dF( P(t,T))
&=& \frac{1}{P(t,T)} dP(t,T)\\
&=& m(s,T)ds +  v(s,T)dB(s) - \frac{1}{2}v(t,T)^\intercal v(t,T) d\langle B^2\rangle(s) \\
&=& \Bigr[ m(s,T) - \frac{1}{2}v(t,T)^\intercal v(t,T) \Bigr] ds + v(s,T)dB(s)
\end{eqnarray*}

\[
\log P(t,T) = \log P(0,T) + \int_0^t\Bigr[ m(s,T) - \frac{1}{2}v(t,T)^\intercal v(t,T) \Bigr] ds + \int_0^t v(s,T)dB(s)
\]

よって,
\begin{eqnarray*}
f(t,T) &=& - \frac{\partial \log P(t,T)}{\partial T}\\
&=&  - \frac{\partial \log P(0,T)}{\partial T} + \int_0^t  \Bigr[\frac{\partial v(t,T)}{\partial T} v(t,T) -\frac{\partial m(s,T)}{\partial T}\Bigr] ds -  \int_0^t \frac{\partial v(s,T)}{\partial T} d B(s)\\
&=& f(0,T)+ \int_0^t \alpha(t,T) ds + \int_0^t \sigma(t,T) dB(s)
\end{eqnarray*}
従って,
\[
\alpha (t,T) = v_T(t,T)v(t,T)-m_T(t,T)\,\,\, \sigma(t,T) + v_T(t,T) = 0
\]


\begin{itembox}[l]{問題2}
(4)を証明せよ.
\end{itembox}

一次元Ito process$X(t),X(x)=0$に対し,$\mathcal{E}(X)(t):=\exp(X(t)-\frac{1}{2}\langle X\rangle(t))$によってIto process\,\,$\mathcal{E}(X)$を定義する.\\
また,$Z(t) := X(t) - \frac{1}{2}(t),\,\,\,f(x)=\exp(x)$として,Ito formulaを適用する:
\begin{eqnarray*}
\mathcal{E}(X)(t)&=&f(Z(t))  \,\,\,\,\text{(伊藤公式より)}\\
&=&f(Z(0)) + \int_0^t(\frac{df}{dx})(Z(s))dZ(s) + \frac{1}{2}\int_0^t (\frac{d^2f}{dx^2})(Z(s))d\langle Z \rangle (s)
\end{eqnarray*}
一方,$f(Z(0))=f(x_0)=e^{x_0},\frac{df}{dx}=\exp(x)\,\,\, \frac{d^2f}{dx^2}=\exp(x)$,
\begin{eqnarray*}
d\langle Z \rangle(s) &=& d\Bigr\langle X(\cdot) - \frac{1}{2}\langle X \rangle(\cdot) \Bigr\rangle(s)\\
&=& d\Bigr\langle X(\cdot) - \frac{1}{2}\langle X \rangle(\cdot), X(\cdot) - \frac{1}{2}\langle X \rangle(\cdot) \Bigr\rangle(s)\\
&=& d\Bigr(
\langle X,X \rangle (s) - \frac{1}{2}\langle \langle X \rangle(\cdot), X \rangle(s) - \frac{1}{2}\langle X, \langle X \rangle(\cdot) \rangle(s) + \frac{1}{4}\langle \langle X \rangle(\cdot), \langle X \rangle(\cdot) \rangle(s) \Bigr)\\
&=& d\langle X,X \rangle(s)\\
&=& d\langle X \rangle(s)
\end{eqnarray*}
以上より,
\begin{eqnarray*}
\mathcal{E}(X)(t)&=& e^{x_0} + \int_0^t \exp(Z(s))(dX(s) - \frac{1}{2}d\langle X \rangle(s)) + \frac{1}{2}\int_0^t \exp(Z(s))d\langle X \rangle(s) \\
&=& e^{x_0} + \int_0^t \exp(Z(s))dX(s)\\
&=& e^{x_0} + \int_0^t \mathcal{E}(s)dX(s)
\end{eqnarray*}
$x_0=0$ので,$\mathcal{E}(X)(t) = \int_0^t \mathcal{E}(X)(s)dX(s) +1$\\
また,\[\alpha(t) := (\mu_t^S(X(t)) -\mu_t^U(X(t)))^\intercal \cdot {(\sigma_t(X(t)))^{-1}}^\intercal\]
\[
X(t):=\int_0^t \alpha(s) d B_U(s)
\]と置いてやると,
\begin{eqnarray*}
\mathcal{E}(X)(t) &=& \mathcal{E}(\int_0^t \alpha(s) d B_U(s) )\\
&=& \exp (\int_0^t\alpha(s)dB_U(S) - \frac{1}{2}\langle X \rangle(t))\\
&=& \exp (\int_0^t(\mu_t^S(X(t)) -\mu_t^U(X(t)))^\intercal \cdot {(\sigma_t(X(t)))^{-1}}^\intercal dB_U(S) - \frac{1}{2}\langle X \rangle(t))\\
\end{eqnarray*}
一方,$\langle X \rangle(t) = \langle X, X \rangle(t) = \int_0^t (K^X(s))^2 ds = \int_0^t \alpha(s)^2 dS$\\
\[
\xi(t) = \mathcal{E}(X)(t) =  \xi(0) + \int_0^t \xi(s)dX(s) =  \xi(0) + \int_0^t \alpha(s)\xi(s)dB_U(s)
\]

\begin{itembox}[l]{問題3}
定理3.1.2より,$T$-claim $H$の価格は$\pi_t(H) = P(t,T)E^{Q^T}[H|\mathcal{F}_t]$で与えられる,この事を証明せよ.
\end{itembox}

ここで,$P(t,T)$は時刻$t$における満期は$T$のzero-coupon bondの価格である.Numeraireを$N$とするとき,一般価格公式は
\[
\pi_t(H) = N(t)E^{Q^N}[\frac{H}{N(T)}|\mathcal{F}_t]
\]
であるので,満期$T$の時,$N(t)=P(t,T),N(T)=P(T,T)=1$を代入すると
\[
\pi_t(H) = N(t)E^{Q^T}[\frac{H}{N(T)}|\mathcal{F}_t] = P(t,T)E^{Q^T}[H|\mathcal{F}_t]
\]
\end{document}