\documentclass[a4paper,11pt]{jsarticle}
\usepackage{graphicx}
\usepackage{wrapfig}
\usepackage{amsmath,amssymb,amsthm}
\usepackage{amssymb}
\usepackage{ascmac}
\usepackage{subfigure}
\usepackage{bm}
\usepackage{setspace}
\usepackage{cases}%左かっこつけるときに必要だった
\usepackage{leftidx}%行列表示用?
\usepackage{fancyhdr}
\usepackage{graphicx}
\usepackage{float}
\usepackage{booktabs}
\usepackage{url}
\usepackage{bm}
\usepackage{verbatim}
\usepackage{calc} 

\setlength{\headsep}{5mm}
\setlength{\oddsidemargin}{-0.5zw} %→にズラす
\setlength{\textheight}{37\baselineskip}
\addtolength{\textheight}{\topskip}
\setlength{\topmargin}{-10mm}
\setlength{\textwidth}{45zw} %文章の幅
\setlength{\textheight}{215mm}
\setlength{\parindent}{1zw}%箇条書きの一文字下げ
\pagestyle{fancy}

\newtheorem{theorem}{定理}
\newtheorem{prop}[theorem]{命題}
\newtheorem{lemma}[theorem]{補題}
\newtheorem{cor}[theorem]{系}
\newtheorem{example}[theorem]{例}
\newtheorem{definition}[theorem]{定義}
\newtheorem{rem}[theorem]{注意}
\newtheorem{guide}[theorem]{参考}
\renewcommand{\proofname}{証明}

\numberwithin{theorem}{section}  % 定理番号を「定理2.3」のように印刷
\numberwithin{equation}{section} % 式番号を「(3.5)」のように印刷
\newcommand{\argmax}{\mathop{\rm arg~max\,}\limits}
\newcommand{\argmin}{\mathop{\rm arg~min\,}\limits}
\newcommand{\st}{\mathop{\rm subject~to\,}\limits}
\newcommand{\sign}{\mathop{\rm sign\,}\limits}

\newcommand{\dom}{\mathop{\mathrm{\bf dom}}\nolimits}
\newcommand{\minimize}{\mathop{\mathrm{\rm minimize}}\limits}
\newcommand{\maximize}{\mathop{\mathrm{\rm maximize}}\limits}

\newcommand{\prox}{\mathop{\mathrm{\rm Prox}}\limits}

\title{2012年度金融リスクマネジメント・第一回レポート}
\author{12M42340 チョウ シホウ}
\date{}

\begin{document}
\pagenumbering{roman}
%\tableofcontents
%\cleardoublepage
\pagenumbering{arabic}
\renewcommand{\thepart}{\arabic{part}}

\maketitle

\section{問題1}
$RFS(t,\tau,S_{\alpha,\beta}(t))=0$となる様な$S_{\alpha,\beta}(t)$を定義する.
\begin{equation}
S_{\alpha,\beta}(t) = \frac{P(t,T_{\alpha})-P(t,T_{\beta})}{\sum_{i=\alpha}^{\beta-1} (T_{i+1}-T_{i})P(t,T_{i+1})}
\end{equation}

European Payer Swaptionとは,所有者が固定金利払い、変動金利受けのスワップ取引を行う権利である.

$T_{\alpha}$に於ける,Payer-Swapの価値は
\begin{eqnarray}
&&\sum_{i=\alpha}^{\beta+1} P(t,T_{i+1})(T_{i+1} - T_{i})(L(t,T_i)-K)\\
&=& (S_{\alpha,\beta}(t)-K) \sum_{i=\alpha}^{\beta+1} (T_{i+1} - T_{i})P(t,T_{i+1})\label{PSVALUE}
\end{eqnarray}
なお,
\[
\sum_{i=\alpha}^{\beta+1} (T_{i+1} - T_{i})P(t,T_{i+1}) \ge 0
\]
が成り立つので,\eqref{PSVALUE}の値の正負が$(S_{\alpha,\beta}(t)-K)$によって決める.$(S_{\alpha,\beta}(t)-K)>0$の時,権利を行使すれば,収益が正,$(S_{\alpha,\beta}(t)-K)<0$の時,権利を行使すれば,収益が負.
所有者は権利を行うかどうか選ばれるので,
\[
\begin{cases}
S_{\alpha,\beta}(T_{\alpha}) - K >0 & \text{権利を行使}\\
S_{\alpha,\beta}(T_{\alpha}) - K >0 & \text{何でもしない}
\end{cases}
\]
という行動をとる.
\newpage

\section{問題2}
\begin{proof}
$f:\mathbb{R}\times\mathbb{R}^d \to \mathbb{R}:C^2\text{-function},B(t)=(B^1(t),\dots,B^d(t))$に対して,過程$f(t,B(t))=f(t,B^1(t),\dots,B^d(t))$をito-formulaで求め,微分形式で表す.ただし,$B(t)=(x^{(1)},\dots,x^{(d)})(t)$.
\begin{equation}
df(t,B(t)) = \frac{\partial f}{\partial t} dt + \sum_{i=1}^d \frac{\partial f}{\partial x^{(i)}} d B^{(i)}(t) + \frac{1}{2}\sum_{i=1,j=1}^d \frac{\partial^2 f}{\partial x^{(i)} \partial x^{(j)}} d \langle x^{(i)}, x^{(j)}\rangle (t)
\label{BITO}
\end{equation}
$B(t)$はBrownian motionであるので,
\[
\langle x^{(i)}, x^{(j)}\rangle (t) = \begin{cases}
0 & i \neq j\\
t & i = j
\end{cases}
\]
式\eqref{BITO}に代入すると,
\begin{equation}
df(t,B(t)) = \Bigr( \frac{\partial f}{\partial t}+ \frac{1}{2}\sum_{i=1}^d \frac{\partial^2 f}{\partial (x^{(i)})^2} \Bigr)dt +  \sum_{i=1}^d \frac{\partial f}{\partial x^{(i)}} d B^{(i)}(t)
\label{BITO2}
\end{equation}
式\eqref{BITO2}を両辺積分を取る.
\begin{equation}
f(t,B(t)) = f(0,B(i)) + \int_0^t \Bigr( \frac{\partial f}{\partial t}+ \frac{1}{2}\sum_{i=1}^d \frac{\partial^2 f}{\partial (x^{(i)})^2} \Bigr)dt + \sum_{i=1}^d \int_0^t  \frac{\partial f}{\partial x^{(i)}} d B^{(i)}(t)
\label{BITO3}
\end{equation}
式\eqref{BITO3}の右辺第三項を$H(s)$とおくと
\begin{equation}
H(s) = \frac{\partial f}{\partial s}+ \frac{1}{2}\sum_{i=1}^d \frac{\partial^2 f}{\partial (x^{(i)})^2}
\end{equation}
明らかに,次の式と同じ形式である.
\begin{equation}
X(t) = X(0) + \int_0^t H(s)ds + \sum_{i=1}^d \int_0^t K_i(s)dB^{(i)}(s)
\end{equation}
命題2.10.1より,$H(s)=0$ので,
\begin{equation}
H(s) = \frac{\partial f}{\partial s}+ \frac{1}{2}\sum_{i=1}^d \frac{\partial^2 f}{\partial (x^{(i)})^2} = 0
\end{equation}
\end{proof}

\end{document}