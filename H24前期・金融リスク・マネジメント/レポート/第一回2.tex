\documentclass[a4paper,11pt]{jsarticle}
\usepackage{graphicx}
\usepackage{wrapfig}
\usepackage{amsmath,amssymb,amsthm}
\usepackage{amssymb}
\usepackage{ascmac}
\usepackage{subfigure}
\usepackage{bm}
\usepackage{setspace}
\usepackage{cases}%左かっこつけるときに必要だった
\usepackage{leftidx}%行列表示用?
\usepackage{fancyhdr}
\usepackage{graphicx}
\usepackage{float}
\usepackage{booktabs}
\usepackage{url}
\usepackage{bm}
\usepackage{verbatim}
\usepackage{calc} 

\setlength{\headsep}{5mm}
\setlength{\oddsidemargin}{-0.5zw} %→にズラす
\setlength{\textheight}{37\baselineskip}
\addtolength{\textheight}{\topskip}
\setlength{\topmargin}{-10mm}
\setlength{\textwidth}{45zw} %文章の幅
\setlength{\textheight}{215mm}
\setlength{\parindent}{1zw}%箇条書きの一文字下げ
\pagestyle{fancy}

\newtheorem{theorem}{定理}
\newtheorem{prop}[theorem]{命題}
\newtheorem{lemma}[theorem]{補題}
\newtheorem{cor}[theorem]{系}
\newtheorem{example}[theorem]{例}
\newtheorem{definition}[theorem]{定義}
\newtheorem{rem}[theorem]{注意}
\newtheorem{guide}[theorem]{参考}
\renewcommand{\proofname}{証明}

\numberwithin{theorem}{section}  % 定理番号を「定理2.3」のように印刷
\numberwithin{equation}{section} % 式番号を「(3.5)」のように印刷
\newcommand{\argmax}{\mathop{\rm arg~max\,}\limits}
\newcommand{\argmin}{\mathop{\rm arg~min\,}\limits}
\newcommand{\st}{\mathop{\rm subject~to\,}\limits}
\newcommand{\sign}{\mathop{\rm sign\,}\limits}

\newcommand{\dom}{\mathop{\mathrm{\bf dom}}\nolimits}
\newcommand{\minimize}{\mathop{\mathrm{\rm minimize}}\limits}
\newcommand{\maximize}{\mathop{\mathrm{\rm maximize}}\limits}

\newcommand{\prox}{\mathop{\mathrm{\rm Prox}}\limits}

\title{2012年度金融リスクマネジメント・第一回レポート}
\author{12M42340 チョウ シホウ}
\date{}

\begin{document}
\pagenumbering{roman}
%\tableofcontents
%\cleardoublepage
\pagenumbering{arabic}
\renewcommand{\thepart}{\arabic{part}}

\maketitle

\section{問題1}
\section{問題2}
\begin{proof}
$f:\mathbb{R}\times\mathbb{R}^d \to \mathbb{R}:C^2\text{-function},B(t)=(B^1(t),\dots,B^d(t))$に対して,過程$f(t,B(t))=f(t,B^1(t),\dots,B^d(t))$をito-formulaで求め,微分形式で表す.ただし,$B(t)=(x^{(1)},\dots,x^{(d)})(t)$.
\begin{equation}
df(t,B(t)) = \frac{\partial f}{\partial t} d\mathcal{A} + \sum_{i}^d \frac{\partial f}{\partial x^{(i)}} d B^{(i)}(\mathcal{A}) + \frac{1}{2}\sum_{i,j}^d \frac{\partial^2 f}{\partial x^{(i)} \partial x^{(j)}} d \langle x^{(i)}, x^{(j)}\rangle (\mathcal{A})
\label{BITO}
\end{equation}
$B(t)$はBrownian motionであるので,$d \langle x^{(i)}, x^{(j)}\rangle (\mathcal{A}) = d\mathcal{A}$,式\eqref{BITO}に代入すると,
\begin{equation}
df(t,B(t)) = \Bigr( \frac{\partial f}{\partial \mathcal{A}}+ \frac{1}{2}\sum_{i}^d \frac{\partial^2 f}{\partial (x^{(i)})^2} \Bigr)d\mathcal{A} +  \sum_{i}^d \frac{\partial f}{\partial x^{(i)}} d B^{(i)}(\mathcal{A})
\label{BITO2}
\end{equation}
式\eqref{BITO2}を両方積分を取る.
\begin{equation}
f(t,B(t)) = f(0,B(0)) + \int_0^t \Bigr( \frac{\partial f}{\partial \mathcal{A}}+ \frac{1}{2}\sum_{i}^d \frac{\partial^2 f}{\partial (x^{(i)})^2} \Bigr)d\mathcal{A} + \sum_{i}^d \int_0^t  \frac{\partial f}{\partial x^{(i)}} d B^{(i)}(\mathcal{A})
\label{BITO3}
\end{equation}
式\ref{BITO3}の右辺第二項を$H(\mathcal{A})$とおくと
\begin{equation}
H(\mathcal{A}) = \frac{\partial f}{\partial \mathcal{A}}+ \frac{1}{2}\sum_{i}^d \frac{\partial^2 f}{\partial (x^{(i)})^2}
\end{equation}
明らかに,次の式と同じ形式である.
\begin{equation}
X(t) = X(0) + \int_0^t H(\mathcal{A})d\mathcal{A} + \sum_{i=1}^d \int_0^t K_i(\mathcal{A})dB^{(i)}(\mathcal{A})
\end{equation}
命題2.10.1より,$H(\mathcal{A})=0$ので,
\begin{equation}
H(\mathcal{A}) = \frac{\partial f}{\partial \mathcal{A}}+ \frac{1}{2}\sum_{i}^d \frac{\partial^2 f}{\partial (x^{(i)})^2} = 0
\end{equation}
\end{proof}

\end{document}