\documentclass[a4paper,11pt]{jsarticle}
\usepackage{graphicx}
\usepackage{wrapfig}
\usepackage{amsmath,amssymb,amsthm}
\usepackage{amssymb}
\usepackage{ascmac}
\usepackage{subfigure}
\usepackage{bm}
\usepackage{setspace}
\usepackage{cases}%左かっこつけるときに必要だった
\usepackage{leftidx}%行列表示用?
\usepackage{fancyhdr}
\usepackage{graphicx}
\usepackage{float}
\usepackage{booktabs}
\usepackage{url}
\usepackage{bm}
\usepackage{verbatim}
\usepackage{calc} 

\setlength{\headsep}{5mm}
\setlength{\oddsidemargin}{-0.5zw} %→にズラす
\setlength{\textheight}{37\baselineskip}
\addtolength{\textheight}{\topskip}
\setlength{\topmargin}{-10mm}
\setlength{\textwidth}{45zw} %文章の幅
\setlength{\textheight}{215mm}
\setlength{\parindent}{1zw}%箇条書きの一文字下げ
\pagestyle{fancy}

\newtheorem{theorem}{定理}
\newtheorem{prop}[theorem]{命題}
\newtheorem{lemma}[theorem]{補題}
\newtheorem{cor}[theorem]{系}
\newtheorem{example}[theorem]{例}
\newtheorem{definition}[theorem]{定義}
\newtheorem{rem}[theorem]{注意}
\newtheorem{guide}[theorem]{参考}
\renewcommand{\proofname}{証明}

\numberwithin{theorem}{section}  % 定理番号を「定理2.3」のように印刷
\numberwithin{equation}{section} % 式番号を「(3.5)」のように印刷
\newcommand{\argmax}{\mathop{\rm arg~max\,}\limits}
\newcommand{\argmin}{\mathop{\rm arg~min\,}\limits}
\newcommand{\st}{\mathop{\rm subject~to\,}\limits}
\newcommand{\sign}{\mathop{\rm sign\,}\limits}

\newcommand{\dom}{\mathop{\mathrm{\bf dom}}\nolimits}
\newcommand{\minimize}{\mathop{\mathrm{\rm minimize}}\limits}
\newcommand{\maximize}{\mathop{\mathrm{\rm maximize}}\limits}

\newcommand{\prox}{\mathop{\mathrm{\rm Prox}}\limits}

\lhead{ノルムのFenchel双対と汎用解法}
\chead{}
\rhead{輪読}
\lfoot{チョウ シホウ}
\cfoot{\thepage}
\rfoot{6月25日}
\renewcommand{\footrulewidth}{0.4pt}
\title{ノルムのFenchel双対と汎用解法}
\author{12M42340 チョウ シホウ}
\date{6月25日}

\begin{document}
\pagenumbering{roman}
%\tableofcontents
%\cleardoublepage
\pagenumbering{arabic}
\renewcommand{\thepart}{\arabic{part}}

%\thispagestyle{empty}
\section{前回に引き続き}
\subsection{正則化付き最適化問題の最適性条件(再掲)}
\begin{equation}
\minimize_{\bm{w} \in \mathbb{R}^p}\,\,\, f(\bm{w}) + \lambda \Omega(\bm{w}) \label{PP}
\end{equation}
に対し,あるベクトル$\bm{w} \in \mathbb{R}^p$が最適解となる必要十分条件は以下の式を満たす$-\frac{1}{\lambda} \nabla f(\bm{w}) \in \partial \Omega(\bm{w})$である.
\begin{equation}
\partial \Omega(\bm{w}) = \begin{cases}
\{ \bm{z} \in \mathbb{R}^p \,\,\,|\,\,\, \Omega^*(\bm{z}) \le 1 \} & \text{\,\,\,if\,\,\,} \bm{w} = 0 \\
\{ \bm{z} \in \mathbb{R}^p \,\,\,|\,\,\, \Omega^*(\bm{z}) \le 1 \text{\,\,\,\,and\,\,\,\,} \bm{z}^\intercal \bm{w} = \Omega(\bm{w}) \}  & \text{\,\,\,otherwise.}
\end{cases}
\label{OptCond}
\end{equation}

次のFenchel Conjugate of a NormとFenchel-Young Inequalityを用いて式(\ref{OptCond})を証明できる.
\begin{itembox}[l]{Fenchel Conjugate of a Norm}
\begin{prop}
Let $\Omega$ be a norm on $\mathbb{R}^p$.The following equality holds for any $\bm{z} \in \mathbb{R}^p$
\begin{equation}
\sup_{\bm{w}\in \mathbb{R}^p}[\bm{z}^\intercal\bm{w} - \Omega(\bm{w})] = l_{\Omega^*(\bm{z})} = \begin{cases}
0 &\,\,\,\, \Omega^*(\bm{z}) \le 1 \\
+ \infty & \,\,\,\, \text{otherwise.}
\end{cases}
\end{equation}
\end{prop}
\end{itembox}
\begin{itembox}[l]{Fenchel-Young Inequality}
\begin{prop}
$\bm{w} \in \mathbb{R}^p,f \in \mathbb{R}^p,\bm{z} \in \dom f^* \neq \emptyset$とおくと,次の不等式
\[
f(\bm{w}) + f^*(\bm{z}) \ge \bm{w}^\intercal \bm{z}
\]が成り立ち,また,$\bm{z} \in \partial f(\bm{w})$の時等号が成立.
\end{prop}
\end{itembox}
証明は板書で.

\subsection{ノルムに対してのFenchel共役の双対性}
前回引き続いて,次の不等式の双対性を示す.
\begin{equation}
\max_{\bm{z} \in \mathbb{R}^p,\Omega^*(\bm{z}) \le \lambda} -f^*(\bm{z}) \le \min_{\bm{w}\in \mathbb{R}^p} f(\bm{w}) + \lambda \Omega(\bm{w})
\end{equation}
関数$f$の定義域が空でないとき等号が成り立つ.\\
弱双対性はFenchel-Young Inequalityより証明され,強双対性は本\cite{borwein}の定理3.3.5を参照.板書.

\subsection{双対性を用いて問題(\ref{PP})を解く}
$z(\bm{w}^*) = \nabla f(\bm{w}^*)$は双対問題の唯一の解である.一般的には,$\bm{z}=\min(1,\frac{\lambda}{\Omega^*(\nabla f (\bm{w}))})\nabla f(\bm{w})$のとうな$\bm{z}$を取り,$\bm{z}(\bm{w}^*)$を最適解して双対ギャップをゼロにする.


\subsection{Design Matrixを用いた定式化}
関数$f$を関数$\psi:\mathbb{R}^n \to \mathbb{R}$,Design Matrix $\bm{X}$を用いて$f(\bm{w}) = \psi(\bm{Xw})$と変換する.
\begin{eqnarray}
\min_{\bm{w} \in \mathbb{R}^p,\bm{u} \in \mathbb{R}^n} &\,\,\,\,& \psi(\bm{u}) + \lambda \Omega(\bm{w}) \\
\st &\,\,\,\,& \bm{u} = \bm{X}\bm{w}
\end{eqnarray}

また,この問題をラグランジュ乗数$\bm{\alpha}^\intercal  \in \mathbb{R}^n$を用いてラグランジュ双対問題に変換すると
\begin{eqnarray}
\min_{\bm{w}\in\mathbb{R}^p,\bm{u} \in \mathbb{R}^n} \,\,\, \max_{\bm{\alpha}^\intercal  \in \mathbb{R}^n} && \psi(\bm{u}) + \lambda \Omega(\bm{w}) + \lambda \bm{\alpha}^\intercal  (\bm{X}\bm{w}-\bm{u}) \\
\min_{\bm{w}\in\mathbb{R}^p,\bm{u} \in \mathbb{R}^n} \,\,\, \max_{\bm{\alpha}^\intercal  \in \mathbb{R}^n} && (\psi(\bm{u}) - \lambda\bm{\alpha}^\intercal \bm{u}) + \lambda (\Omega(\bm{w}) +   \bm{\alpha}^\intercal \bm{X}\bm{w})
\end{eqnarray}
そして,Fenchel双対問題は
\begin{eqnarray}
\max_{\bm{\alpha} \in \mathbb{R}^n} -\psi^*(\lambda \bm{\alpha}) \text{\,\,\,\,such that\,\,\,\,} \Omega^*(\bm{X}^\intercal  \bm{\alpha}) \le \lambda
\end{eqnarray}

\section{汎用メソッド}

\begin{equation}
\minimize_{\bm{w} \in \mathbb{R}^p}\,\,\,\, f(\bm{w}) + \lambda \Omega(\bm{w})
\label{eq2.1}
\end{equation}
損失関数$f$と正則化項$\Omega$が同時に凸である場合だけで式(\ref{eq2.1})は凸である.

\subsection{劣勾配法}
もし問題の劣勾配を効率よく算出できれば,すべての制約なし凸問題が劣勾配法で解ける.式(\ref{eq2.1})に対して,損失関数$f$と正則化項$\Omega$の劣勾配が計算できれば劣勾配法で解ける.

その反復アルゴリズムは:

\begin{equation}
\bm{w}_{t+1} = \bm{w}_t - \frac{\alpha}{t}(\bm{s} + \lambda\bm{s}'), \,\,\, \text{ where } \bm{s} \in \partial f(\bm{w}_t), \bm{s}' \in \partial \Omega (\bm{w}_t) 
\end{equation}

本\cite{iloco}より,この反復アルゴリズムが大域的収束,収束率は$F(\bm{w}_t) - \min_{\bm{w}\in \mathbb{R}^p} F(\bm{w}) = O(\frac{1}{\sqrt{t}})$,だが,実際に劣勾配法の収束スピードが遅い,スパース解を求めるのも難しい.

\subsection{LP,QP,SOCP,SDP問題に定着}
この一章すべての正則化付き最適化問題(正則化付き最小二乗問題)はSDP問題またはQPなどSDPより簡単な問題に定着できる.

例えば,次の問題
\begin{equation}
\min_{\bm{w} \in \mathbb{R}^p} \frac{1}{2n}\| \bm{y} - \bm{X}\bm{w} \|_2^2 + \lambda \Omega(\bm{w})
\end{equation}
の実数制約を非負制約に変形すると,次のQP問題に定着できる
\begin{equation}
\min_{\bm{w}_+,\bm{w}_- \in \mathbb{R}^p_+} \frac{1}{2n}\| \bm{y} - \bm{X}\bm{w}_+ + \bm{X}\bm{w}_-  \|_2^2 + \lambda \Omega(1^\intercal \bm{w}_+ - 1^\intercal \bm{w}_-)
\end{equation}
汎用ツールを使用すれば高精度解(小さい双対ギャップ)を求められるが,機械学習にとっては効率悪くて贅沢すぎる.なぜなら,
\begin{enumerate}
\item これらのツールが問題の特徴を無視しスピードが遅い,メモリー不足で落ちる可能性もある.
\item Bottou and Bousquet(2007)より,機械学習に対して高精度の解がいらない. 
\end{enumerate}

\begin{comment}
\section{近接法(Proximal Methods)}

前回もこのアルゴリズムを簡単に説明したが,今回はより早いバージョンを紹介する.5/28の第四節Shrinking/Thresholding for Regularized Optimizationを参照してください.

\subsection{原理}
近接法は式(\ref{eq2.1})のような問題(リプシッツ連続の微分可能関数$f$と微分不可能の$\lambda \Omega$の和)を解くの専用アルゴリズムであり,その収束率と大規模非スムーズに対する性能に焦点が当てられている.

近接法は次の式に定義される.

\begin{equation}
\min_{\bm{w} \in \mathbb{R}^p} f(\bm{w}^\intercal ) + \nabla f(\bm{w}^\intercal )^\intercal (\bm{w}-\bm{w}^\intercal ) + \lambda \Omega(\bm{w}) + \frac{L}{2}\|\bm{w} - \bm{w}^\intercal \|_2^2 \label{PROX_P}
\end{equation}

ただし,二次項$\frac{L}{2}\|\bm{w} - \bm{w}^\intercal \|_2^2$は近接項(proximal term)という.$L>0$は$f$のリプシッツ定数であり,$\nabla f$の上界を制約する.式(\ref{PROX_P})は前回レジメの$\alpha = \frac{1}{L}$バージョン.

近接法のアプローチは,反復ごとに$f$を現在点に線形近接し,近接項を利用して$\bm{w}^\intercal $の近傍を更新する.
\begin{equation}
\min_{\bm{w} \in \mathbb{R}^p} \frac{1}{2}\| \bm{w} - \bigr(\bm{w}^\intercal  - \frac{1}{L}\nabla f(\bm{w}^\intercal )\bigr) \|_2^2 + \frac{\lambda}{L} \Omega(\bm{w});
\end{equation}
次の式に書き直す
\begin{equation}
\min_{\bm{w}\in\mathbb{R}^p} \frac{1}{2}\| \bm{u} - \bm{w} \|^2 + \mu \Omega(\bm{w}); \label{PROX_N}
\end{equation}

非スムーズ正則化項$\Omega$がなければ,式(\ref{PROX_N})は次の劣勾配法とは同じである.
\[ \bm{w}^{t+1} \Rightarrow \bm{w}^\intercal  - \frac{1}{L} \]

\begin{equation}
\prox_{\frac{\lambda}{L} \Omega} \Bigr( \bm{w}^\intercal  - \frac{1}{L}\nabla f(\bm{w}^\intercal ) \Bigr)
\end{equation}

\begin{equation}
f(\bm{w}^*) \le M_f^L(\bm{w}^\intercal , \bm{w}_L^*) := f(\bm{w}^\intercal ) + \nabla f(\bm{w}^\intercal )^\intercal (\bm{w}_L^*-\bm{w}^\intercal ) + \frac{L}{2}\| \bm{w}_L^* - \bm{w}^2 \|_2^2
\end{equation}
\end{comment}

\begin{thebibliography}{99}
\bibitem{bach} Francis Bach, Rodolphe Jenatton, EJulien Mairal and Guillaume Obozinski (2012) "Optimization with Sparsity-Inducing Penalties", 
Foundations and Trends in Machine Learning: Vol. 4: No 1, pp 1-106. 
\bibitem{fukushima} 福島雅夫, 非線形最適化の基礎, 朝倉書店, 2001
\bibitem{borwein}  Borwein, Jonathan; Lewis, Adrian (2006). Convex Analysis and Nonlinear Optimization: Theory and Examples (2 ed.). Springer. ISBN 978-0-387-29570-1.
\bibitem{iloco} Y. Nesterov, Introductory Lectures on Convex Optimization: A Basic Course (Applied Optimization), 1st ed.    Springer Netherlands.
\bibitem{nips-t} Steve Wright, NIPS Tutorial, 6 December 2010,\,\,\url{ http://pages.cs.wisc.edu/~swright/nips2010/sjw-nips10.pdf}
\end{thebibliography}
\end{document}