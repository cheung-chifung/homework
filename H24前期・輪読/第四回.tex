\documentclass[a4paper,11pt]{jsarticle}
\usepackage{graphicx}
\usepackage{wrapfig}
\usepackage{amsmath,amssymb,amsthm}
\usepackage{amssymb}
\usepackage{ascmac}
\usepackage{subfigure}
\usepackage{bm}
\usepackage{setspace}
\usepackage{cases}%左かっこつけるときに必要だった
\usepackage{leftidx}%行列表示用?
\usepackage{fancyhdr}
\usepackage{graphicx}
\usepackage{float}
\usepackage{booktabs}
\usepackage{url}
\usepackage{bm}
\usepackage{verbatim}
\usepackage{calc}
\usepackage{algorithm}
\usepackage{algorithmic}

\setlength{\headsep}{5mm}
\setlength{\oddsidemargin}{-0.5zw} %→にズラす
\setlength{\textheight}{37\baselineskip}
\addtolength{\textheight}{\topskip}
\setlength{\topmargin}{-10mm}
\setlength{\textwidth}{45zw} %文章の幅
\setlength{\textheight}{215mm}
\setlength{\parindent}{1zw}%箇条書きの一文字下げ
\pagestyle{fancy}

\newtheorem{theorem}{定理}
\newtheorem{prop}[theorem]{命題}
\newtheorem{lemma}[theorem]{補題}
\newtheorem{cor}[theorem]{系}
\newtheorem{example}[theorem]{例}
\newtheorem{definition}[theorem]{定義}
\newtheorem{rem}[theorem]{注意}
\newtheorem{guide}[theorem]{参考}
\renewcommand{\proofname}{証明}

\numberwithin{theorem}{section}  % 定理番号を「定理2.3」のように印刷
\numberwithin{equation}{section} % 式番号を「(3.5)」のように印刷
\newcommand{\argmax}{\mathop{\rm arg~max\,}\limits}
\newcommand{\argmin}{\mathop{\rm arg~min\,}\limits}
\newcommand{\st}{\mathop{\rm subject~to\,}\limits}
\newcommand{\sign}{\mathop{\rm sign\,}\limits}

\newcommand{\dom}{\mathop{\mathrm{\bf dom}}\nolimits}
\newcommand{\minimize}{\mathop{\mathrm{\rm minimize}}\limits}
\newcommand{\maximize}{\mathop{\mathrm{\rm maximize}}\limits}

\newcommand{\prox}{\mathop{\mathrm{\rm Prox}}\limits}

\lhead{Proximal operators of mixed norms}
\chead{}
\rhead{輪読}
\lfoot{チョウ シホウ}
\cfoot{\thepage}
\rfoot{7月30日}
\renewcommand{\footrulewidth}{0.4pt}
\title{Proximal operators of mixed norms}
\author{12M42340 チョウ シホウ}
\date{7月30日}

\begin{document}
\pagenumbering{roman}
%\tableofcontents
%\cleardoublepage
\pagenumbering{arabic}
\renewcommand{\thepart}{\arabic{part}}

%\thispagestyle{empty}

\section{Hierarchical $\ell_1$/$\ell_q$-norms}
\begin{equation}
\min_{\bm{w} \in \mathbb{R}^p}\,\,\,\, \frac{1}{2} \|\bm{u} - \bm{w}\| + \mu \Omega(\bm{w})
\end{equation}
\begin{equation}
\max_{\bm{v} \in \mathbb{R}^p}\,\,\,\, -\frac{1}{2}\bigr[ \|\bm{u} - \bm{v}\|_2^2 - \|\bm{u}\|^2 \bigr] \text{\,\,\,\,\,\,such that\,\,\,\,}\Omega^*(\bm{v})\le \mu
\end{equation}

\begin{itembox}[l]{Proximal operator of Hierarchical norms}
次のノルムを考え:
$\Omega:\bm{w} \to \sum_{g\in\mathcal{G}}\|\bm{w}_g\|_q,\text{\,\,\,\,with\,\,\,\,}q\in\{2,\infty\}$,$\mathcal{G}$は木構造のグループ集合である.$\preceq$記号を次のように定義する.
\[
g_1 \preceq g_2 \Rightarrow \{ g_1 \subseteq g_2 \text{\,\,\,or\,\,\,} g_1 \cap g_2 = \emptyset  \}
\]
$g_1 \preceq \dots \preceq g_m,\,\,m=|\mathcal{G}|$,
$\prox_g := \bm{w}_g \mapsto \prox_{\mu\|\cdot\|_q}(\bm{w}_g)$とおくと,Hierarchical $\ell_1$/$\ell_q$-normsのproximal operatorは
\[
\prox_{\mu\Omega}=\prox_{g_m}\circ\dots\circ\prox_{g_1}
\]
\end{itembox}
\begin{lemma} $\ell_1,\ell_q$-ノルムの双対ノルムは$\ell_\infty.\ell_{q^*}$-ノルム,$\frac{1}{q}+\frac{1}{q^*}=1$
\end{lemma}
\begin{lemma} {\bf 双対問題と最適解条件\\}
双対問題
\begin{equation}
\max_{\bm{\xi}\in \mathbb{R}^{p \times |\mathcal{G}|}} -\frac{1}{2}\Bigr(\| \bm{u} - \sum_{g\in\mathcal{G}} \bm{\xi}^g \|_2^2 - \|\bm{u}\|_2^2 \Bigr)  \text{\,\,\,s.t.\,\,}\forall g\in\mathcal{G},\|\bm{\xi}^g\|_* \le \mu \text{\,\,and\,\,\,}\bm{\xi}_j^g=0 \text{\,\,if\,\,}j\neq g
\end{equation}

最適解条件
\begin{equation}
\begin{cases}
\bm{w} = \bm{u} - \sum_{g\in \mathcal{G}} \bm{\xi}^g \\
\forall g \in \mathcal{G},\,\,\, \bm{\xi}^g = \Pi_{\mu}^*(\bm{w}_g + \bm{\xi}^g)
\end{cases}
\end{equation}
\end{lemma}

\begin{algorithm}[H]
\caption{Block coordinate descent algorithm}
\label{A1}
\textbf{\,\,input:\,\,\,\,}$\bm{u}\in\mathbb{R}^p$とグループ集合$\mathcal{G}$\\
\textbf{\,\,Outputs:\,\,\,\,}$(\bm{w},\bm{\xi})$(主双対最適解)\\
\textbf{\,\,Initialization:\,\,\,\,}$\bm{w}=\bm{u}, \bm{\xi}=0$\\
\,\,WHILE (maximum number of iterations not reached) DO\\
\,\,\,\,\,\,FOR $g\in\mathcal{G}$ DO\\
\,\,\,\,\,\,\,\,\,\,$\bm{w} \leftarrow \bm{u}- \sum_{h\neq g} \bm{\xi}^h$\\
\,\,\,\,\,\,\,\,\,\,$\bm{\xi}^g \leftarrow \Pi_{\mu}^*(\bm{w}_g)$\\
\,\,\,\,\,\,END FOR\\
\,\,END WHILE\\
\,\,$\bm{w} \leftarrow \bm{u} - \sum_{g\in \mathcal{G}} \bm{\xi}^g$
\end{algorithm}

\begin{lemma}
Alogorithm 1で,任意の$g\in \mathcal{G},h\in \mathcal{G},g \preceq h$に対し,もし$\bm{\xi}^g$を$\bm{\xi}^h$の前に更新すれば,$\bm{\xi}^h$は$\bm{\xi}^g$の最適解条件に影響を与えない.
\end{lemma}

\begin{algorithm}[H]
\caption{Practical Computation of the Proximal Operator for $\ell_2$- or $\ell_\infty$-norms.}
\label{A1}
\textbf{\,\,input:\,\,\,\,}$\bm{u}\in\mathbb{R}^p$と木構造グループ集合$\mathcal{G}$\\
\textbf{\,\,Outputs:\,\,\,\,}$(\bm{w})$(主問題最適解)\\
\textbf{\,\,Initialization:\,\,\,\,}$\bm{w}=\bm{u}$\\
\,\,\,\,FOR $g\in\mathcal{G}$,following the order $\preceq$ DO\\
\,\,\,\,\,\,\,\,$\bm{w}_g \leftarrow \bm{w}_g - \Pi_{\|\cdot\|_* \le \mu}(\bm{w}_g) = \prox_{\mu \|\cdot\|_q} (\bm{w}_g)$\\
\,\,\,\,END FOR
\end{algorithm}
\section{Combined $\ell_1+\ell_1$/$\ell_q$-norm(sparse group Lasso)}
\begin{itembox}[l]{group Lasso}
データサンプルは$\bm{x}_j \in \mathbb{R}^m,j=1,\dots,n$,辞書(Dictionary)は$\bm{D}\in \mathbb{R}^{m \times p},\bm{D}=[\bm{d}_1,\dots,\bm{d}_p]$,$\bm{x}_j$は$\bm{x}_j = \bm{D}\bm{\alpha}_j + \epsilon$と表せる,$\alpha \in \mathbb{R}^p$は辞書の線形結合重み,$\epsilon \in \mathbb{R}^m$はノイズである.この問題のgroup Lasso解法は
\begin{equation}
\min_{\bm{a}} \frac{1}{2}\|\bm{x}_j - \bm{D}\bm{\alpha}\|_2^2 + \mu \Omega_{\mathcal{G}}(\bm{\alpha})
\end{equation}
\end{itembox}
この問題に対するのHierarchical Lassoは
\begin{equation}
\min_{\bm{a}} \frac{1}{2}\|\bm{x}_j - \bm{D}\bm{\alpha}\|_2^2 + \mu_1 \Omega_{\mathcal{G}}(\bm{\alpha}) + \mu_2 \|\bm{\alpha}\|_1
\end{equation}

\section{Overlapping $\ell_1$/$\ell_\infty$-norms}
overlapがあるが木構造のないグラーフ集合のProximal operatorを算出するのは難しいだが,$q=\infty$だけは効率的に計算できる.

\begin{thebibliography}{99}
\bibitem{jenatton} R. Jenatton, J. Mairal, G. Obozinski, and F. Bach, "Proximal methods for sparse hierarchical dictionary learning," Proc. ICML, 2010. 
\bibitem{jenatton} Proximal Methods for Sparse Hierarchical Dictionary Learning: Supplementary Materials\\\url{http://www.di.ens.fr/~jenatton/paper/SupplementaryMaterialsICML2010.pdf}
\bibitem{l1lqball} S. Sra, Fast Projections onto $\ell_{1,q}$-Norm Balls for Grouped Feature Selection Lecture Notes in Computer Science, 2011, Volume 6913, Machine Learning and Knowledge Discovery in Databases, Pages 305-317

\end{thebibliography}
\end{document}