\documentclass{jsarticle}
\topmargin=-3.0cm \oddsidemargin=0.1cm \evensidemargin=0.1cm
\textwidth=16 true cm \textheight=25 true cm

\usepackage{otf}
\usepackage{amsmath,amssymb}

\begin{document}
\title{2012年度数理計画法特論・第一回レポート}
\author{{\normalsize 社会理工学研究科 経営工学専攻 チョウ シホウ 学籍番号 12M42340}}
\date{}
\maketitle


\baselineskip 0.7 cm

{\bf 問題1}
\begin{enumerate}
\item 線形相補問題は次の問題とは同じ:
\begin{eqnarray}
\text{min \,\,\,\,\,} && z(Mz+q)\\
\text{s.t. \,\,\,\,\,} && Mz +q \ge 0\\
&& z \ge 0
\end{eqnarray}
この最適化問題は混合整数線形問題(MILP)に定着できる:
\begin{eqnarray}
\text{min \,\,\,\,\,} && a\\
\text{s.t. \,\,\,\,\,} && 0 \le a(Mz+q) \le k \label{milp1}\\
&& 0 \le az \le e-k \label{milp2}\\
&&0 \le a \le 1  \\
&&e = (1,1,\dots,1)^T \in \mathbb{R}^n \\
&& k \in \{0,1\}^n
\end{eqnarray}
$\forall i = 1,2,\dots,n,k_i \in \{0,1\},0 \le az_i \le 1-k_i,0 \le a(Mz+q)_i \le k_i$の制約がある。\\
最適解$(z^*,k^*,a^*)$が存在すれば、それが線形相補性条件を満たすことを示す。
\begin{enumerate}
\item $a* > 0$の時:$k_i^*=1$ならば(\ref{milp2})により$z_i=0$、$k_i^*=0$ならば(\ref{milp1})により$(Mz+q)_i=0$、両方でも$z_i^*(Mz+q)_i=0$を満たす。
\item $a^*=0$の時:$z=0$と$Mz+q=0$が同時に成り立たないので、$k_i^*=1$ならば$(Mz+q)_i>0,z^*_i=0$、$k_i^*=0$ならば$(Mz+q)_i =0, z^*_i>0$、同じく$z_i^*(Mz+q)_i=0$を満たす。
\end{enumerate}
最適解があればすべての$i$に対して$z_i^*(Mz+q)_i=0$が成立なので、線形相補性条件$z(Mz+q)=\sum_{i}^n z_i(Mz+q)_i =0$を満たす。

\item 
ジョブ$j$がジョブ$k$より優先ならば$y_{jk}=1$、そうではなければ$y_{jk}=0$。$M$は大きい人工変数。
\begin{eqnarray}
\text{min \,\,\,\,\,} && \sum_{j=1}^n w_jC_j\\
\text{s.t. \,\,\,\,\,}&& C_j \ge p_j, \forall j \in \{1,2,..,n\} \label{c_a}\\
&& C_j \ge 0, \forall j \in \{1,2,..,n\}  \label{c_b} \\
&& C_j + p_k \le C_k + M(1-y_{jk}), \forall j,k \in \{1,2,..,n\} , j<k \label{c_1} \\
&& C_k + p_j \le C_j + My_{jk},  \forall j,k \in \{1,2,..,n\} , j<k \label{c_2} \\
&& y_{jk} \in \{0,1\}, \forall j,k \in \{1,2,..,n\} , j<k
\end{eqnarray}
制約条件(\ref{c_1})(\ref{c_2})により
\begin{enumerate}
\item $y_{jk}=1$ならば、ジョブ$j$がジョブ$k$より優先、$C_j + p_k \le C_k$
\item $y_{jk}=0$ならば、ジョブ$k$がジョブ$j$より優先、$C_k + p_j \le C_j$
\end{enumerate}
制約条件(\ref{c_a})(\ref{c_b})と組み合わせすると、最適解が存在すれば、
\[
\text{ジョブ$j$の完了時刻:}C_j = p_j + \sum_{a,b \in \{1,2,..,n\} ,a < b} y_{ab}p_b
\]
が必ず成立。
\item 
区分線形関数$f$は単調増加であれば、$\lambda_i$変数で区分線形関数を区分によりモデル化できる。$k_i$が区分$i$の上界以上であれば$\lambda_i=1$、そうではなければ$\lambda_i=0$。
\begin{eqnarray}
\text{min \,\,\,\,\,}&& c^Tx+y_i \\
\text{s.t. \,\,\,\,\,}&& Ax=b \\
&&x \ge 0\\
&&x_1 = \sum_{i=1}^m k_i \label{dsplit}\\
&& y_1 = \sum_{i=1}^{m-1} \lambda_i k_i \frac{f(p_{i+1})-f(p_{i})}{p_{i+1}-p_i} \label{ddelta}\\
&& \lambda_{i+1} \ge \lambda_i, \forall i \in \{1,2,\dots,m-1\} \label{dorder}\\
&& \lambda_i \in \{0,1\}, \forall i \in \{1,2,\dots,m-1\} \\
&& k_i \le \lambda_i(p_{i+1}- p_{i}), \forall i \in \{1,2,\dots,m-1\} \label{do1} \\
&& \lambda_{i-1}(p_{i+1} - p_{i}) \le k_i, \forall i \in \{2,3,\dots,m \} \label{do2}
\end{eqnarray}
\end{enumerate}
制約条件(\ref{dsplit})により、$x_1$を$m$個の$k_i$に分割できる、(\ref{ddelta})により、$f(p_i)$と$\lambda_i$の組み合わせで$f(x_1)$を表せる。(\ref{dorder})(\ref{dorder})(\ref{dorder})により、$\lambda_{i-1}(p_{i+1} - p_{i}) \le k_i \le \lambda_i(p_{i+1}- p_{i})$、$k_i$が$x_i$の区分に定められる。

{\bf 問題2}

主問題(1)を問題文の双対問題のような形に変換する、$I$は単位行列である。
\begin{equation*}
\begin{split}
\text{max\,\,\,\,\,}& c^Tx \\
\text{s.t.\,\,\,\,\,}&
\begin{bmatrix}
A^T  & I
\end{bmatrix}
^Tx \le
\begin{bmatrix}
b \\
e
\end{bmatrix}
\\
&x \ge 0
\end{split}
\end{equation*}
すると、その双対性を使って次の双対問題に等価変換できる。
\begin{equation*}
\begin{split}
\text{min\,\,\,\,\,}&
\begin{bmatrix}
b \\
e
\end{bmatrix}^T
\begin{bmatrix}
p \\
q
\end{bmatrix}
\\
\text{s.t.\,\,\,\,\,}&
\begin{bmatrix}
A^T  & I
\end{bmatrix}
\begin{bmatrix}
p \\
q
\end{bmatrix}
\ge
c
\\
& \begin{bmatrix}
p \\
q
\end{bmatrix} \ge 0
\end{split}
\end{equation*}
この問題は下記のと同じである。
\begin{equation*}
\begin{split}
\text{min\,\,\,\,\,}& b^Tp + e^Tq \\
\text{s.t.\,\,\,\,\,}&
A^Tp + q \ge c \\
& p \ge 0, q \ge 0
\end{split}
\end{equation*}



{\bf 問題3}\\
元問題の制約を1-tree制約に緩和する。\\
\begin{center}
\begin{tabular}{c| c c c c c c}
& $a$ & $b$ & $c$ & $d$ & $e$ & $f$ \\
\hline
$a$ & $0$ & $12$ & $7$ & $13$ & $5$ & $4$ \\
$b$ & $12$ & $0$ & $6$ & $9$ & $7$ & $11$ \\
$c$ & $7$ & $6$ & $0$ & $4$ & $2$ & $7$ \\
$d$ & $13$ & \textcircled{$9$} & \textcircled{$4$} & $0$ & $6$ & $10$ \\
$e$ & \textcircled{$5$} & $7$ & \textcircled{$2$} & $6$ & $0$ & $7$ \\
$f$ & \textcircled{$4$} & \textcircled{$11$} & $7$ & $10$ & $7$ & $0$ \\
\end{tabular}
\end{center}
上界:巡回路になるという条件の元で、枝の重みが小さいものから順に加える。その閉路は$a \rightarrow f \rightarrow b \rightarrow d \rightarrow c \rightarrow e \rightarrow a$、枝の重みの総和は$4+11+9+4+2+5=35$\\
\begin{center}
\begin{tabular}{c| c c c c c c}
& $a$ & $b$ & $c$ & $d$ & $e$ & $f$ \\
\hline
$a$ & $0$ & $12$ & $7$ & $13$ & $5$ & $4$ \\
$b$ & $12$ & $0$ & \textcircled{$6$} & $9$ & $7$ & $11$ \\
$c$ & $7$ & $6$ & $0$ & $4$ & $2$ & $7$ \\
$d$ & $13$ & $9$ & \textcircled{$4$} & $0$ & $6$ & $10$ \\
$e$ & \textcircled{$5$} & $7$ & \textcircled{$2$} & $6$ & $0$ & $7$ \\
$f$ & \textcircled{$4$} & $11$ & \textcircled{$7$} & $10$ & $7$ & $0$ \\
\end{tabular}
\end{center}
下界:一つの頂点を固定して1-treeを作る。頂点$a$を選び、残ったグラフで枝の重みの総和が最小の全域木は$(b,c),(c,f),(c,e),(c,d)$であり、頂点$a$に接続する重みが最小の二つ枝$(a,e),(a,f)$をその全域木に加える。下界は$(6+7+2+4)+(4+5)=28$


{\bf 問題4}\\
\begin{eqnarray}
\text{min \,\,\,\,\,} && c^Tx \\
\text{s.t. \,\,\,\,\,} && A_ix \le b_i \,\,\,\,\, (i = 1,2,\dots,k) \\
&& x \in \mathbb{Z}_+^n
\end{eqnarray}
元問題の$c=\sum_{i=1}^k c_i$であり、$x=\sum_{i=1}^k x_i$とおくと
\begin{eqnarray}
c^Tx &=& (c_1 + c_2 + \dots + c_k)^Tx\\
&=& (c_1^T + c_2^T + \dots + c_k^T)(x_1 + x_2 + \dots + x_k)\\
&=& \sum_{i=1}^k\sum_{j=1}^k c_ix_j
\end{eqnarray}
なお、制約条件$A_ix \le b_i, \,\,\, (i=1,2,\dots,k)$は
\begin{eqnarray*}
&A_1x_1 + A_1x_2 + \dots + A_1x_k \le b_1& \\
&A_2x_1 + A_2x_2 + \dots + A_2x_k \le b_2 &\\
&\vdots&\\
&A_kx_1 + A_kx_2 + \dots + A_kx_k \le b_k &\\
\end{eqnarray*}
すると、元問題は次の問題とは同じである。
\begin{eqnarray}
\text{min \,\,\,\,\,} && \sum_{i=1}^k \sum_{j=1} c_i^Tx_j \\
\text{s.t. \,\,\,\,\,} && A_i\sum_{j=1}^k x_j \le b_i \,\,\,\,\, (i = 1,2,\dots,k) \\
&& x_i \in \mathbb{Z}_+^n \,\,\,\,\, (i = 1,2,\dots,k)
\end{eqnarray}
この問題に二つの制約条件を追加すれば、新しい問題ができる。
\begin{eqnarray}
\text{min \,\,\,\,\,} && \sum_{i=1}^k \sum_{j=1} c_i^Tx_j \\
\text{s.t. \,\,\,\,\,} && A_i\sum_{j=1}^k x_j \le b_i \,\,\,\,\, (i = 1,2,\dots,k) \\
&& A_ix_j =0 \,\,\,\,\, (\forall i\in \{1,2,\dots,k\}, j\in \{1,2,\dots,k\};i \neq j) \\
&& c_ix_j =0 \,\,\,\,\, (\forall i\in \{1,2,\dots,k\}, j\in \{1,2,\dots,k\};i \neq j) \\
&& x_i \in \mathbb{Z}_+^n \,\,\,\,\, (i = 1,2,\dots,k)
\end{eqnarray}
次のように等価変換できる。
\begin{eqnarray}
\text{min \,\,\,\,\,} && \sum_{i=1}^k c_i^Tx_i \\
\text{s.t. \,\,\,\,\,} && A_ix_i \le b_i \,\,\,\,\, (i = 1,2,\dots,k) \\
&& x_i \in \mathbb{Z}_+^n \,\,\,\,\, (i = 1,2,\dots,k)
\end{eqnarray}
変数$x_i$は互いに独立であるので、この緩和問題は変数$x_i$ごとに独立して$k$個の整数計画問題に分解でき、元問題より解きやすい、ある意味の緩和問題である。元問題の最適解が存在する十分必要条件KKT条件は
\begin{eqnarray}
\begin{cases}
\alpha_i \ge 0 \,\,\,\,\, (i = 1,2,\dots,k)\\
b_i - A_i\sum_{j=1}^k x_j \ge 0 \,\,\,\,\, (i = 1,2,\dots,k) \\
\alpha_i^T(b_i - A_i\sum_{j=1}^k x_j) =0 \,\,\,\,\, (i = 1,2,\dots,k) \\
\sum_{j=1}^k c_j^T - \sum_{i=1}^k \alpha_i^T A_i =0 \\
x_i \in \mathbb{Z}_+^n \,\,\,\,\, (i = 1,2,\dots,k)
\end{cases}
\end{eqnarray}
緩和問題のKKT条件は
\begin{eqnarray}
\begin{cases}
\alpha_i \ge 0 \,\,\,\,\, (i = 1,2,\dots,k)\\
b_i - A_ix_i \ge 0 \,\,\,\,\, (i = 1,2,\dots,k) \\
\alpha_i^T(b_i - A_i x_j) =0 \,\,\,\,\, (i = 1,2,\dots,k) \\
c_i^T - \alpha_i^T A_i =0 \,\,\,\,\, (i = 1,2,\dots,k) \\
A_ix_j =0 \,\,\,\,\, (\forall i\in \{1,2,\dots,k\}, j\in \{1,2,\dots,k\};i \neq j) \\
c_ix_j =0 \,\,\,\,\, (\forall i\in \{1,2,\dots,k\}, j\in \{1,2,\dots,k\};i \neq j) \\
x_i \in \mathbb{Z}_+^n \,\,\,\,\, (i = 1,2,\dots,k)
\end{cases}
\end{eqnarray}
明らかに、緩和問題の最適解$x_i^*$が存在すれば元問題の最適解$x^*=\sum_{i=1}^k x_i$も存在する、逆は成立しない、元問題の実行可能解集合$X$と緩和問題の実行可能解集合$\bar{X}$の関係は$\bar{X} \subset X$。\\
%一方、変数$x_i$ごとに独立する整数計画問題$XP_i$
%\begin{eqnarray}
%\text{min \,\,\,\,\,} && c_i^Tx_i \\
%\text{s.t. \,\,\,\,\,} && A_ix_i \le b_i \\
%&& x_i \in \mathbb{Z}_+^n
%\end{eqnarray}
%を考えるみると、$x_i>0,x=\sum_{i=}^k x_i$ので、元問題が最適解$x_i^*$があれば、任意の個々の整数計画問題$XP_i$の最適解$x_i^*<x^*$である、また、問題$XP_i$が実行可能ならば、元問題も実行可能である
\end{document}
