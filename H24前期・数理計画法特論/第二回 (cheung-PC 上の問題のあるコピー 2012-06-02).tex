\documentclass[a4paper,11pt]{jsarticle}
\usepackage{graphicx}
\usepackage{wrapfig}
\usepackage{amsmath,amssymb,amsthm}
\usepackage{amssymb}
\usepackage{ascmac}
\usepackage{subfigure}
\usepackage{bm}
\usepackage{setspace}
\usepackage{cases}%左かっこつけるときに必要だった
\usepackage{leftidx}%行列表示用?
\usepackage{fancyhdr}
\usepackage{graphicx}
\usepackage{float}
\usepackage{booktabs}
\usepackage{url}
\usepackage{bm}

\setlength{\headsep}{5mm}
\setlength{\oddsidemargin}{-0.5zw} %→にズラす
\setlength{\textheight}{37\baselineskip}
\addtolength{\textheight}{\topskip}
\setlength{\topmargin}{-10mm}
\setlength{\textwidth}{45zw} %文章の幅
\setlength{\textheight}{215mm}
\setlength{\parindent}{1zw}%箇条書きの一文字下げ
\pagestyle{fancy}

\newtheorem{theorem}{定理}
\newtheorem{prop}[theorem]{命題}
\newtheorem{lemma}[theorem]{補題}
\newtheorem{cor}[theorem]{系}
\newtheorem{example}[theorem]{例}
\newtheorem{definition}[theorem]{定義}
\newtheorem{rem}[theorem]{注意}
\newtheorem{guide}[theorem]{参考}
\renewcommand{\proofname}{証明}

\numberwithin{theorem}{section}  % 定理番号を「定理2.3」のように印刷
\numberwithin{equation}{section} % 式番号を「(3.5)」のように印刷
\newcommand{\argmax}{\mathop{\rm arg~max\,}\limits}
\newcommand{\argmin}{\mathop{\rm arg~min\,}\limits}
\newcommand{\st}{\mathop{\rm subject~to\,}\limits}
\newcommand{\sign}{\mathop{\rm sign\,}\limits}

\newcommand{\dom}{\mathop{\mathrm{\bf dom}}\nolimits}
\newcommand{\minimize}{\mathop{\mathrm{\rm minimize}}\limits}


\lhead{2012年度数理計画特論・第二回レポート}
\chead{}
\rhead{12M42340}
\lfoot{チョウ シホウ}
\cfoot{\thepage}
\rfoot{6月11日}
\renewcommand{\footrulewidth}{0.4pt}
\title{2012年度数理計画特論・第二回レポート}
\author{12M42340 チョウ シホウ}
\date{6月11日}

\begin{document}
\pagenumbering{roman}
%\tableofcontents
%\cleardoublepage
\pagenumbering{arabic}
\renewcommand{\thepart}{\arabic{part}}

%\thispagestyle{empty}

\section{問題1:解答}

\begin{enumerate}
\item 問題文のラグランジュ問題
\[
\begin{split}
\minimize &\,\,\,\,  \sum_{i \in M} \sum_{j \in N} c_{ij}x_{ij} + \sum_{j \in N} f_j y_j + \sum_{i \in M} u_i(1-\sum_{j\in N}x_ij)\\
\text{subject to} &\,\,\,\,  x_{ij} \le y_j \,\, (i \in M, j \in N),\\
&\,\,\,\,  x_{ij} \ge 0  \,\, (i \in M, j \in N),\\
&\,\,\,\,  y_j \in \{0,1\} \,\, (j \in N)
\end{split}
\]
を次のように$N$個の子問題に割り当てられる.
子問題($j$)
\[
\begin{split}
\minimize &\,\,\,\, \sum_{i \in M} c_{ij}x_{ij} + f_jy_j - \sum_{i \in M} u_i x_{ij} \\
\text{subject to} &\,\,\,\,  x_{ij} \le y_j \,\, (i \in M, j \in N),\\
&\,\,\,\,  x_{ij} \ge 0  \,\, (i \in M, j \in N),\\
&\,\,\,\,  y_j \in \{0,1\} \,\, (j \in N)
\end{split}
\]
{\bf 子問題$1$:}
\[
\begin{split}
\minimize &\,\,\,\, (3-6)x_{11} + (6-3)x_{21} + (4-5)x_{31} + (6-4)x_{41} +8y_1\\
\text{subject to} &\,\,\,\,  x_{i1} \le y_1 \,\, (i \in M),\\
&\,\,\,\,  x_{i1} \ge 0  \,\, (i \in M),\\
&\,\,\,\,  y_1 \in \{0,1\}
\end{split}
\]
\begin{itemize}
\item $y_1 = 0$の時,$\forall i \in M, x_{i1}=0$,最小値$0$である.
\item $y_1 = 1$の時,$\forall i \in M, 0 \le x_{i1} \le 1,x_{11} = \min \{-3,0\} = -3, x_{21} = \min \{3,0\} = 0, x_{31} = \min \{-1,0\} = -1, x_{41} = \min \{2,0\} = 0$,最小値は$-3+0-1+0+8=4$である.
\end{itemize}
従って,子問題$1$の最小値は$\min \{ 0,4 \} = 0$,最適解は$x_{11}=x_{21}=x_{31}=_{41}=0,y_1=0$である.同様に,子問題$2,3$はそれぞれ:\\\\
{\bf 子問題$2$}
\[
\begin{split}
\minimize &\,\,\,\, (8-6)x_{12} + (2-3)x_{22} + (1-5)x_{32} + (2-4)x_{42} +5y_1\\
\text{subject to} &\,\,\,\,  x_{i2} \le y_2 \,\, (i \in M),\\
&\,\,\,\,  x_{i2} \ge 0  \,\, (i \in M),\\
&\,\,\,\,  y_2 \in \{0,1\}
\end{split}
\]
最適解は$x_{12}=0,x_{22}=-1,x_{32}=-4,x_{42}=-2,y_2=1$,最適値は$-2$である.\\

{\bf 子問題$3$}
\[
\begin{split}
\minimize &\,\,\,\, (1-6)x_{13} + (3-3)x_{23} + (8-5)x_{33} + (5-4)x_{43} +9y_1\\
\text{subject to} &\,\,\,\,  x_{i3} \le y_3 \,\, (i \in M),\\
&\,\,\,\,  x_{i3} \ge 0  \,\, (i \in M),\\
&\,\,\,\,  y_3 \in \{0,1\}
\end{split}
\]
最適解は$x_{13}=x_{23}=x_{33}=_{43}=0,y_3=0$,最適値は$0$である\\

従って,緩和問題の最適値は$0-2+0=-2$である.


\item
\item
\item
\end{enumerate}

\section{問題2:解答}
次の線形計画問題
\begin{equation}
\begin{split}
\minimize &\,\,\,\, \bm{c}^T\bm{x}\\
\text{subject to} &\,\,\,\,  \bm{Ax} \ge \bm{b}, \\
&\,\,\,\,  \bm{Cx} = \bm{d}\\
&\,\,\,\,  \bm{x} \ge \bm{0}
\end{split}
\end{equation}
とその双対問題
\begin{equation}
\begin{split}
\minimize &\,\,\,\, \bm{b}^T\bm{y_1} + \bm{d}^T\bm{y_2}\\
\text{subject to} &\,\,\,\,  \bm{A}^T \bm{y_1} + \bm{C}^T \bm{y_2} \le \bm{c} \\
&\,\,\,\,  \bm{y_1} \ge \bm{0},\\
&\,\,\,\,  \bm{y_2} \text{は実数}
\end{split}
\end{equation}
を考え.次のラグランジュ緩和をとると
\begin{equation}
L()
\end{equation}


\section{問題3:解答}
\section{問題4:解答}
\begin{enumerate}
\item
整数非負問題
\[
X=\{ y \in \mathbb{Z}_+^4 | 4y_1 + 5y_2 + 9y_3 + 12y_4 \le 34 \}
\]
とするとき,不等式$y_2+y_3+2y_4 > 6$が成り立つと仮定すると.
\[
y_2,y_3,y_4 \in \mathbb{Z}_+^4,\,\, 5y_2 + 5y_3 + 10y_4 \ge 31
\]
\item
\end{enumerate}




\end{document}