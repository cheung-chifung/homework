\documentclass[a4paper,11pt]{jsarticle}
\usepackage{graphicx}
\usepackage{wrapfig}
\usepackage{amsmath,amssymb,amsthm}
\usepackage{amssymb}
\usepackage{ascmac}
\usepackage{subfigure}
\usepackage{bm}
\usepackage{setspace}
\usepackage{cases}%左かっこつけるときに必要だった
\usepackage{leftidx}%行列表示用?
\usepackage{fancyhdr}
\usepackage{graphicx}
\usepackage{float}
\usepackage{booktabs}
\usepackage{url}
\usepackage{bm}
\usepackage{verbatim}
\usepackage{calc} 

\setlength{\headsep}{5mm}
\setlength{\oddsidemargin}{-0.5zw} %→にズラす
\setlength{\textheight}{37\baselineskip}
\addtolength{\textheight}{\topskip}
\setlength{\topmargin}{-10mm}
\setlength{\textwidth}{45zw} %文章の幅
\setlength{\textheight}{215mm}
\setlength{\parindent}{1zw}%箇条書きの一文字下げ
\pagestyle{fancy}

\newtheorem{theorem}{定理}
\newtheorem{prop}[theorem]{命題}
\newtheorem{lemma}[theorem]{補題}
\newtheorem{cor}[theorem]{系}
\newtheorem{example}[theorem]{例}
\newtheorem{definition}[theorem]{定義}
\newtheorem{rem}[theorem]{注意}
\newtheorem{guide}[theorem]{参考}
\renewcommand{\proofname}{証明}

\numberwithin{theorem}{section}  % 定理番号を「定理2.3」のように印刷
\numberwithin{equation}{section} % 式番号を「(3.5)」のように印刷
\newcommand{\argmax}{\mathop{\rm arg~max\,}\limits}
\newcommand{\argmin}{\mathop{\rm arg~min\,}\limits}
\newcommand{\st}{\mathop{\rm subject~to\,}\limits}
\newcommand{\sign}{\mathop{\rm sign\,}\limits}

\newcommand{\dom}{\mathop{\mathrm{\bf dom}}\nolimits}
\newcommand{\minimize}{\mathop{\mathrm{\rm minimize}}\limits}
\newcommand{\maximize}{\mathop{\mathrm{\rm maximize}}\limits}

\lhead{2012年度数理計画特論・第二回レポート}
\chead{}
\rhead{12M42340}
\lfoot{チョウ シホウ}
\cfoot{\thepage}
\rfoot{6月11日}
\renewcommand{\footrulewidth}{0.4pt}
\title{2012年度数理計画特論・第二回レポート}
\author{12M42340 チョウ シホウ}
\date{6月11日}

\begin{document}
\pagenumbering{roman}
%\tableofcontents
%\cleardoublepage
\pagenumbering{arabic}
\renewcommand{\thepart}{\arabic{part}}

%\thispagestyle{empty}

\section{問題1:解答}
\begin{enumerate}
\item 問題文のラグランジュ問題
\[
\begin{split}
\minimize &\,\,\,\,  \sum_{i \in M} \sum_{j \in N} c_{ij}x_{ij} + \sum_{j \in N} f_j y_j + \sum_{i \in M} u_i(1-\sum_{j\in N}x_{ij})\\
\text{subject to} &\,\,\,\,  x_{ij} \le y_j \,\, (i \in M, j \in N),\\
&\,\,\,\,  x_{ij} \ge 0  \,\, (i \in M, j \in N),\\
&\,\,\,\,  y_j \in \{0,1\} \,\, (j \in N)
\end{split}
\]
を次のように$N$個の子問題に割り当てられる.
子問題($j$)
\[
\begin{split}
\minimize &\,\,\,\, \sum_{i \in M} c_{ij}x_{ij} + f_jy_j - \sum_{i \in M} u_i x_{ij} \\
\text{subject to} &\,\,\,\,  x_{ij} \le y_j \,\, (i \in M, j \in N),\\
&\,\,\,\,  x_{ij} \ge 0  \,\, (i \in M, j \in N),\\
&\,\,\,\,  y_j \in \{0,1\} \,\, (j \in N)
\end{split}
\]
{\bf 子問題$1$:}
\[
\begin{split}
\minimize &\,\,\,\, (3-6)x_{11} + (6-3)x_{21} + (4-5)x_{31} + (6-4)x_{41} +8y_1 \\
\text{subject to} &\,\,\,\,  x_{i1} \le y_1 \,\, (i \in M),\\
&\,\,\,\,  x_{i1} \ge 0  \,\, (i \in M),\\
&\,\,\,\,  y_1 \in \{0,1\}
\end{split}
\]
\begin{itemize}
\item $y_1 = 0$の時,$\forall i \in M, x_{i1}=0$,最小値$0$である.
\item $y_1 = 1$の時,$\forall i \in M, 0 \le x_{i1} \le 1, $ 
\[\begin{split}
\min \{-3,0\} = -3 \Rightarrow x_{11} = 0, &\,\,\, \min \{3,0\} = 0 \Rightarrow x_{21} = 0,\\ \min \{-1,0\} = -1 \Rightarrow x_{31} = 0, &\,\,\, \min \{2,0\} = 0 \Rightarrow x_{41} = 0,
\end{split}\]
最小値は$-3+0-1+0+8=4$である.
\end{itemize}
従って,子問題$1$の最小値は$\min \{ 0,4 \} = 0$,最適解は$x_{11}=x_{21}=x_{31}=_{41}=0,y_1=0$である.同様に,子問題$2,3$はそれぞれ:\\\\
{\bf 子問題$2$}
\[
\begin{split}
\minimize &\,\,\,\, (8-6)x_{12} + (2-3)x_{22} + (1-5)x_{32} + (2-4)x_{42} +5y_1\\
\text{subject to} &\,\,\,\,  x_{i2} \le y_2 \,\, (i \in M),\\
&\,\,\,\,  x_{i2} \ge 0  \,\, (i \in M),\\
&\,\,\,\,  y_2 \in \{0,1\}
\end{split}
\]
最適解は$x_{12}=0,x_{22}=1,x_{32}=1,x_{42}=1,y_2=1$,最適値は$-2$である.\\

{\bf 子問題$3$}
\[
\begin{split}
\minimize &\,\,\,\, (1-6)x_{13} + (3-3)x_{23} + (8-5)x_{33} + (5-4)x_{43} +9y_1\\
\text{subject to} &\,\,\,\,  x_{i3} \le y_3 \,\, (i \in M),\\
&\,\,\,\,  x_{i3} \ge 0  \,\, (i \in M),\\
&\,\,\,\,  y_3 \in \{0,1\}
\end{split}
\]
最適解は$x_{13}=x_{23}=x_{33}=_{43}=0,y_3=0$,最適値は$0$である\\

従って,
$\bm{u}$における最適解$\bm{X}^*$は
\[
\bm{X}^* = \begin{pmatrix}
0 & 0   & 0\\
0 & 1 & 0\\
0 & 1 & 0\\
0 & 1 & 0\\
\end{pmatrix}
\]
緩和問題の最適値は$0- 2+0 + \sum_{i \in M} u_i = 16 $である.
\item
上記の問題の最適解$\bm{X}^*$を用いて$\bm{u} = (6,3,5,4)$における劣勾配を算出する.\\
\[
\bm{b} = \begin{pmatrix}
1 \\ 1 \\ 1 \\ 1\\
\end{pmatrix} \,\,\,
\bm{A} = \begin{pmatrix}
1 \\ 1 \\ 1
\end{pmatrix}
\,\,\,\,
\bm{X}^* = \begin{pmatrix}
0 & 0   & 0\\
0 & 1 & 0\\
0 & 1 & 0\\
0 & 1 & 0\\
\end{pmatrix}
\,\,\,
\bm{u} = \begin{pmatrix}
6 \\ 3 \\ 5 \\ 4
\end{pmatrix}
\]に対して,
劣勾配$\bm{g}$は
\[
\bm{g} = \bm{b}-\bm{X}^*\bm{A} = \begin{pmatrix}
1 \\ 1 \\ 1 \\ 1\\
\end{pmatrix} - \begin{pmatrix}
0 & 0   & 0\\
0 & 1 & 0\\
0 & 1 & 0\\
0 & 1 & 0\\
\end{pmatrix} \begin{pmatrix}
1 \\ 1 \\ 1
\end{pmatrix} = \begin{pmatrix}
1 \\ 0 \\ 0 \\ 0
\end{pmatrix}
\]
\item
(FLP2)問題を考え,
\[
\begin{split}
\minimize &\,\,\,\,  \sum_{i \in M} \sum_{j \in N} c_{ij}x_{ij} + \sum_{j \in N} f_j y_j \\
\text{subject to} &\,\,\,\,  \sum_{j\in N} x_{ij} = 1 \,\, (i \in M),\\
&\,\,\,\,  \sum_{i \in M} x_{ij} \le |M|y_j \,\, (j \in N),\\
&\,\,\,\,  x_{ij} \ge 0 \,\, (i \in M, j \in N), \\
&\,\,\,\, y_j \in \{ 0,1 \} \,\, (j \in N),
\end{split}
\]
(FLP1)問題と(FLP2)問題の制約条件の違うところは,
\begin{eqnarray}
\text{for (FLP1)}\,\,\,\,\,\, & x_{ij} \le y_j \,\, &\,\,\,\,\,\, (i \in M, j \in N), \label{FLP1COND}\\
\text{for (FLP2)}\,\,\,\,\,\, & \sum_{ i \in M} x_{ij} \le |M|y_j \,\, &\,\,\,\,\,\, (j \in N), \label{FLP2COND}
\end{eqnarray}
\begin{itemize}
\item $y_j=0$の時\\
$x_{ij}\ge0$ので,$x_{ij} \le 0 \,\, (i \in M) \Leftrightarrow \sum_{i \in M} x_{ij} \le 0$が成り立ち,制約\ref{FLP1COND}と制約\ref{FLP2COND}とは同値である.
\item $y_j=1$の時\\
明らかに
\[
x_{ij} \le 1 \,\, (i \in M) \Rightarrow \sum_{i \in M} x_{ij} \le \sum_{i \in M} 1 \Leftrightarrow \sum_{i \in M} x_{ij} \le |M|
\]が成立,また,共通制約条件$\sum_{j\in N} x_{ij} = 1 \,\, (i \in M) ,\,\,\,x_{ij} \ge 0 \,\, (i \in M, j \in N)$より,$0 \le x_{ij} \le 1$ので,逆向き
\[
\sum_{i \in M} x_{ij} \le |M| \Leftrightarrow  \sum_{i \in M} x_{ij} \le \sum_{i \in M} 1
\Leftrightarrow \sum_{i \in M} (1-x_{ij}) \ge 0 
 \Rightarrow
(1 - x_{ij})\ge 0  \,\, (i \in M) 
\]も成り立つため,両問題の実行可能領域が等しいことが分かられる.

\end{itemize}
\item 制約条件$x_{ij} \le y_j \,\, (i \in M, j \in N),$と$\sum_{ i \in M} x_{ij} \le |M|y_j \,\,  (j \in N)$と$\sum_{j\in N}x_{ij} = 1$をラグランジュ緩和して.(FLP1)と(FLP2)の目的関数はそれぞれ
\[
\begin{split}
\text{(FLP1)}\,
\min &\, \sum_{i \in M} \sum_{j \in N} c_{ij}x_{ij} + \sum_{i \in M} u_i(1-\sum_{j\in N}x_{ij}) + \sum_{j \in N} f_j y_j + \sum_{i \in M} \sum_{j \in N} \lambda_{ij}(x_{ij} - y_j) \\
\text{(FLP2)}\,
\min &\,  \sum_{i \in M} \sum_{j \in N} c_{ij}x_{ij} + \sum_{i \in M} u_i(1-\sum_{j\in N}x_{ij}) + \sum_{j \in N} f_j y_j + \sum_{j \in N} \lambda_j(\sum_{i \in M} x_{ij} - |M|y_j)
\end{split}
\]
である.明らかに,(FLP2)のほうがラグランジュ乗数が少ない.また,両問題を$j$によって子問題を分解すれば,
\[
\begin{split}
\text{(FLP1-SUB-$j$)}\,
\min &\,  \sum_{i \in M}( c_{ij} - u_i + \lambda_{ij})x_{ij} +  (f_j - \sum_{i \in M} \lambda_{ij})y_j \\
\text{(FLP2-SUB-$j$)}\,
\min &\,  \sum_{i \in M}( c_{ij} - u_i )x_{ij} +  \lambda_j|M|(\sum_{i\in M} x_{ij} - y_j) +f_jy_j
\end{split}
\]
子問題(FLP2-SUB-$j$)に対して,問1.1のように,$y_j$を$0$と$1$に固定すれば$\lambda_j$の値も論じられる.だが,子問題(FLP1-SUB-$j$)では,一つの値に対して$x_{ij},\lambda_{ij}$が幾つの組み合わせもあって,あらゆる,子問題(FLP2-SUB-$j$)より実行可能領域が広い\\
例として$\bm{u}=(6,3,5,4)$の$j=1$の子問題を考え,子問題(FLP2-SUB-$1$)は
\[
\minimize \,   -3x_{11} + 3x_{21} - x_{31} + 2x_{41} +  4\lambda_j(\sum_{i\in M} x_{i1} - y_j) + 8y_j
\]
\begin{itemize}
\item $y_j=0$の時,$\forall i \in M, x_{i1}=0$,最小値$0$である.
\item $y_j=1$の時,$x_{11}=x_{31}=1,x_{21}=x_{41}=0$,$\sum_{i\in M} x_{i1} = 2 > y_j$,目的関数を最小化するため,$\lambda_j=0$.
\end{itemize}
一方,子問題(FLP1-SUB-$1$)は
\[
\minimize \,   (-3+\lambda_{11})x_{11} + (3+\lambda_{21})x_{21} + (-1+\lambda_{31})x_{31} + (2+\lambda_{41})x_{41} + (8-\sum_{i\in M} \lambda_{ij})y_j
\]
$y_j=0$の時,例え$\min(-3+\lambda_{11})$に対して, $x_{11}=0$または$\lambda_{11}=3$も取れる,一意な最適解を求めるのは難しいである($y_j=1$も同様).従って,子問題(FLP1-SUB-$j$)と比べ,子問題(FLP2-SUB-$j$)はより簡単に解ける.

\end{enumerate}

\newpage
\section{問題2:解答}

ひとまず,実数$y,\bm{u}_2$を次のように非負制約に変換する,
\[
\begin{split}
y = y_+ - y_-, \bm{u}_2 = \bm{u}_{2+} - \bm{u}_{2-}\\
y_+ \ge 0, \,\, y_- \ge 0,\,\, \bm{u}_{2+} \ge \bm{0}, \bm{u}_{2-} \ge \bm{0}
\end{split}
\]
すると,$\bm{u}_1,\bm{u}_2,y$に関する線形計画問題
\begin{equation}
\begin{split}
\minimize &\,\,\,\,  y \\
\text{subject to} &\,\,\,\, \bm{c}^T\bm{x}^k + \bm{u}_1^T(\bm{b}_1 - \bm{A}_1\bm{x}^k) + \bm{u}_2^T(\bm{b}_2 - \bm{A}_2\bm{x}^k) \le y\,\, \,\,(k = 1,2,\dots,K) \\
&\,\,\,\, \bm{u}_1 \ge 0, \bm{u_2}\text{は実数}
\end{split}
\end{equation}
を書き直す.
\begin{equation}
\begin{split}
\minimize &\,\,\,\,  y_+ - y_- \\
\text{subject to} &\,\,\,\, \bm{u}_1^T(\bm{b}_1 - \bm{A}_1\bm{x}^1) + \bm{u}_{2+}^T(\bm{b}_2 - \bm{A}_2\bm{x}^1) -\bm{u}_{2-}^T(\bm{b}_2 - \bm{A}_2\bm{x}^1) - y_+ + y_- \le \bm{c}^T\bm{x}^1  \\
&\hspace{5cm} \vdots\\
&\,\,\,\, \bm{u}_1^T(\bm{b}_1 - \bm{A}_1\bm{x}^k) + \bm{u}_{2+}^T(\bm{b}_2 - \bm{A}_2\bm{x}^k) -\bm{u}_{2-}^T(\bm{b}_2 - \bm{A}_2\bm{x}^k) - y_+ + y_- \le \bm{c}^T\bm{x}^k  \\
&\,\,\,\, \bm{u}_1 \ge 0, y_+ \ge 0, \,\, y_- \ge 0,\,\, \bm{u}_{2+} \ge \bm{0}, \bm{u}_{2-} \ge \bm{0}
\end{split}
\label{PrimalProb}
\end{equation}
ただし,$\bm{u}_1,\bm{x}^k,\bm{b}_1 \in \mathbb{R}^n, \bm{u}_{2+},\bm{u}_{2-},\bm{b}_2 \in \mathbb{R}^m$\\
そして,標準形に変換する,
\begin{equation}
\bm{A} = 
\begin{pmatrix}
(\bm{b}_1 - \bm{A}_1\bm{x}^1)^T & (\bm{b}_2 - \bm{A}_2\bm{x}^1)^T & -(\bm{b}_2 - \bm{A}_2\bm{x}^1)^T & -1 & 1 \\
(\bm{b}_1 - \bm{A}_1\bm{x}^1)^T & (\bm{b}_2 - \bm{A}_2\bm{x}^1)^T & -(\bm{b}_2 - \bm{A}_2\bm{x}^1)^T &-1 & 1 \\
&\vdots&\vdots&&\\
(\bm{b}_1 - \bm{A}_1\bm{x}^1)^T & (\bm{b}_2 - \bm{A}_2\bm{x}^1)^T & -(\bm{b}_2 - \bm{A}_2\bm{x}^1)^T & -1 & 1
\end{pmatrix}
\in \mathbb{R}^{K} \times \mathbb{R}^{n+2m+2}
\end{equation}
\begin{equation}
\begin{comment}[
\bm{I}_n = \begin{pmatrix}
1\\
1\\
\vdots\\
1
\end{pmatrix} \in \mathbb{R}^n
\,\,\,\
\end{comment}
\bm{X} = \begin{pmatrix}
\bm{u}_1\\
\bm{u}_{2+}\\
\bm{u}_{2-}\\
y_+\\
y_-
\end{pmatrix} \in \mathbb{R}^{n+2m+2} \,\,\,
\bm{B} = \begin{pmatrix}
 \bm{c}^T\bm{x}^1 \\
 \bm{c}^T\bm{x}^2 \\
\vdots\\
 \bm{c}^T\bm{x}^K 
\end{pmatrix}  \in \mathbb{R}^K \,\,\,
\bm{C} = \begin{pmatrix}
0\\
0\\
\vdots\\
0\\
1\\
-1
\end{pmatrix} \in \mathbb{R}^{n+2m+2}\,\,\,
\bm{\Lambda} = \begin{pmatrix}
\lambda_1 \\
\lambda_2 \\
\vdots \\
\lambda_k
\end{pmatrix}
\in \mathbb{R}^K
\end{equation}
次の線形計画問題とその双対性を利用して双対問題を求め,問題\ref{PrimalProb}を標準形に変形して
\begin{equation}
\begin{split}
\minimize &\,\,\,\, \bm{c}^T\bm{x}\\
\text{subject to} &\,\,\,\,  \bm{Ax} \ge \bm{b}, \\
&\,\,\,\,  \bm{Cx} = \bm{d}\\
&\,\,\,\,  \bm{x} \ge \bm{0}
\end{split}
\,\,\,\,\,\,\,
\Longrightarrow
\,\,\,\,\,\,\,
\begin{split}
\maximize &\,\,\,\, \bm{b}^T\bm{y}_1 + \bm{d}^T\bm{y}_2\\
\text{subject to} &\,\,\,\,  \bm{A}^T \bm{y}_1 + \bm{C}^T \bm{y}_2 \le \bm{c} \\
&\,\,\,\,  \bm{y}_1 \ge \bm{0} \\
&\,\,\,\,  \bm{y}_2 \text{は実数} \\
\end{split}
\end{equation}
\begin{equation}
\begin{split}
\minimize_{\bm{X}} &\,\,\,\, \bm{C}^T\bm{X}\\
\text{subject to} &\,\,\,\,  \bm{AX} \ge \bm{B}, \\
&\,\,\,\,  \bm{X} \ge \bm{0}
\end{split}
\,\,\,\,\,\,\,
\Longrightarrow
\,\,\,\,\,\,\,
\begin{split}
\maximize_{\bm{\Lambda}} &\,\,\,\, \bm{B}^T\bm{\Lambda} \\
\text{subject to} &\,\,\,\,  \bm{A}^T \bm{\Lambda} \le \bm{C} \\
&\,\,\,\,  \bm{\Lambda} \ge \bm{0} \\
\end{split}
\end{equation}
変数を代入すると
\begin{equation}
\begin{split}
\maximize_{\bm{\Lambda}} &\,\,\,\, \sum_{k=1}^K \lambda_k \bm{c}^T\bm{x}^k    \\
\text{subject to} &\,\,\,\, \sum_{k=1}^K \lambda_k(\bm{b}_1 - \bm{A}_1\bm{x}^k) \ge \bm{0}\\
&\,\,\,\, \sum_{k=1}^K \lambda_k(\bm{b}_2 - \bm{A}_2\bm{x}^k) \ge \bm{0}\\
&\,\,\,\, \sum_{k=1}^K -\lambda_k(\bm{b}_2 - \bm{A}_2\bm{x}^k) \ge \bm{0}\\
&\,\,\,\, \sum_{k=1}^K -\lambda_k \le 1 \\
&\,\,\,\, \sum_{k=1}^K \lambda_k \le -1 \\
&\,\,\,\, \lambda_k \ge 0, \,\, k = 1,2,\dots,K
\end{split}
\end{equation}
整理すれば,双対問題ができる
\begin{equation}
\begin{split}
\maximize_{\bm{\Lambda}} &\,\,\,\, \sum_{k=1}^K \lambda_k \bm{c}^T\bm{x}^k    \\
\text{subject to} &\,\,\,\, \sum_{k=1}^K \lambda_k(\bm{b}_1 - \bm{A}_1\bm{x}^k) \ge \bm{0}\\
&\,\,\,\, \sum_{k=1}^K \lambda_k(\bm{b}_2 - \bm{A}_2\bm{x}^k) = \bm{0}\\
&\,\,\,\, \sum_{k=1}^K \lambda_k = 1 \,\,\, \lambda_k \ge 0, \,\, k = 1,2,\dots,K
\end{split}
\end{equation}

\begin{comment}
###############################################
次の問題
\begin{equation}
\begin{split}
\minimize &\,\,\,\,  y \\
\text{subject to} &\,\,\,\, \bm{c}^T\bm{x}^k + \bm{u}_1^T(\bm{b}_1 - \bm{A}_1\bm{x}^k) + \bm{u}_2^T(\bm{b}_2 - \bm{A}_2\bm{x}^k) \le y \\
&\,\,\,\, \bm{u}_1 \ge 0, \bm{u_2}\text{は実数}
\end{split}
\end{equation}
は
\begin{equation}
L(\bm{u}_1,\bm{u}_2) = \max \Bigr\{ \bm{c}^T\bm{x}^k + \bm{u}_1^T(\bm{b}_1 - \bm{A}_1\bm{x}^k) + \bm{u}_2^T(\bm{b}_2 - \bm{A}_2\bm{x}^k) \Bigr| \begin{cases}
\bm{u}_1&\ge 0,\\
\bm{u}_2&\text{は実数}, \\
k&= 1,2,\dots,K
\end{cases} \Bigr\}
\end{equation}
とは同値である.また,問題$L(\bm{u}_1,\bm{u}_2) $を$k$ごとに分解すると,$K$個の$L_{\bm{x}^k}$を次のように定義でき.$L(\bm{u}_1,\bm{u}_2) $の最適値は$L(\bm{u}_1^*,\bm{u}_2^*) = \max\{ L_{\bm{x}^k} | k = 1,2,\dots,K \}$である.ただし,$L_{\bm{x}^k}^*$は各子問題の最適値である.
\begin{equation}
\begin{split}
\maximize_{\bm{x}^k} &\,\,\,\,  \bm{c}^T\bm{x}^k \\
\text{subject to} &\,\,\,\, \bm{b}_1 - \bm{A}_1\bm{x}^k \ge 0 \\
&\,\,\,\, \bm{b}_2 - \bm{A}_2\bm{x}^k = 0
\end{split}
\label{PrimalSubProb2}
\end{equation}
%ラグランジュ緩和の性質より問題\ref{PrimalProb}の目的関数値$L_p \le  L(\bm{u}_1,\bm{u}_2) $,最適値$L_p^* = \min_{\bm{u}_1,\bm{u}_2} L(\bm{u_1},\bm{u_2}) $.
そして,問題\ref{PrimalSubProb2}を次のように変形できる.
\begin{equation}
\begin{split}
\minimize_{\bm{x}^k} &\,\,\,\,  \bm{c}^T(-\bm{x}^k) \\
\text{subject to} &\,\,\,\,  \bm{A}_1(-\bm{x}^k) \ge (-\bm{b}_1) \\
&\,\,\,\,  \bm{A}_2(-\bm{x}^k) = (-\bm{b}_2)
\end{split}
\label{PrimalSubProb}
\end{equation}
次の

次の線形計画問題
\begin{equation}
\begin{split}
\minimize &\,\,\,\, \bm{c}^T\bm{x}\\
\text{subject to} &\,\,\,\,  \bm{Ax} \ge \bm{b}, \\
&\,\,\,\,  \bm{Cx} = \bm{d}\\
&\,\,\,\,  \bm{x} \ge \bm{0}
\end{split}
\,\,\,\,\,\,\,
\Leftrightarrow
\,\,\,\,\,\,\,
\begin{split}
\minimize &\,\,\,\, \bm{b}^T\bm{y}_2 + \bm{d}^T\bm{y}_2\\
\text{subject to} &\,\,\,\,  \bm{A}^T \bm{y}_1 + \bm{C}^T \bm{y}_1 \le \bm{c} \\
&\,\,\,\,  \bm{y}_1 \ge \bm{0} \\
&\,\,\,\,  \bm{y}_2 \text{は実数} \\
\end{split}
\end{equation}
とその双対問題
\begin{equation}
\begin{split}
\minimize &\,\,\,\, \bm{b}^T\bm{y_1} + \bm{d}^T\bm{y_2}\\
\text{subject to} &\,\,\,\,  \bm{A}^T \bm{y_1} + \bm{C}^T \bm{y_2} \le \bm{c} \\
&\,\,\,\,  \bm{y_1} \ge \bm{0}
\end{split}
\end{equation}
を考え.次のラグランジュ緩和をとると
######################################################3
\end{comment}

\newpage
\section{問題3:解答}
\[
 \bm{x} = \begin{pmatrix} x_0\\ \bm{x_1} \end{pmatrix},\,\, 
 \bm{y} = \begin{pmatrix} y_0\\ \bm{y_1} \end{pmatrix} \in \mathbb{R}^{1+n} \,\,
 \text{に対して,\,\,}
 \bm{x} \circ \bm{y} = \begin{pmatrix}
 x_0y_0+\bm{x_1}^T\bm{y_1}\\
 x_0 \bm{y_1} + y_0 \bm{x_1}
\end{pmatrix}
\]
\begin{enumerate}
\item
\[
\begin{split}
& \bm{x} \circ ( ( \bm{x} \circ \bm{x}) \circ \bm{y} ) \\
=\,\,\, & 
\begin{pmatrix} x_0\\ \bm{x_1} \end{pmatrix} \circ
\Bigr( \Bigr( \begin{pmatrix} x_0\\ \bm{x_1} \end{pmatrix} \circ
\begin{pmatrix} x_0\\ \bm{x_1} \end{pmatrix} \Bigr) \circ
\begin{pmatrix} y_0\\ \bm{y_1} \end{pmatrix} \Bigr) \\
=\,\,\, &
\begin{pmatrix} x_0\\ \bm{x_1} \end{pmatrix} \circ
\Bigr( 
\begin{pmatrix}
x_0x_0 + \bm{x_1}^T\bm{x_1} \\
x_0\bm{x_1} + x_0\bm{x_1} 
\end{pmatrix} \circ
\begin{pmatrix} y_0\\ \bm{y_1} \end{pmatrix} \Bigr) \\
=\,\,\, &
\begin{pmatrix} x_0\\ \bm{x_1} \end{pmatrix} \circ
\begin{pmatrix}
(x_0x_0 + \bm{x_1}^T\bm{x_1})y_0 + (x_0\bm{x_1} + x_0\bm{x_1})^T\bm{y_1} \\
(x_0x_0 + \bm{x_1}^T\bm{x_1})\bm{y_1} + y_0(x_0\bm{x_1} + x_0\bm{x_1}) \\
\end{pmatrix}\\
=\,\,\, &
\begin{pmatrix} x_0\\ \bm{x_1} \end{pmatrix} \circ
\begin{pmatrix}
(x_0x_0 + \bm{x_1}^T\bm{x_1})y_0 + (x_0\bm{x_1} + x_0\bm{x_1})^T\bm{y_1} \\
(x_0x_0 + \bm{x_1}^T\bm{x_1})\bm{y_1} + y_0(x_0\bm{x_1} + x_0\bm{x_1}) \\
\end{pmatrix}\\
%=\,\,\, &
%\begin{pmatrix}
%x_0(x_0x_0 + \bm{x_1}^T\bm{x_1})y_0 + x_0(x_0\bm{x_1} + x_0\bm{x_1})^T\bm{y_1} + \bm{x_1}^T(x_0x_0 + \bm{x_1}^T\bm{x_1})\bm{y_1} + \bm{x_1}^Ty_0(x_0\bm{x_1} + x_0\bm{x_1})  \\
%x_0(x_0x_0 + \bm{x_1}^T\bm{x_1})\bm{y_1} + x_0y_0(x_0\bm{x_1} + x_0\bm{x_1}) + (x_0x_0 + \bm{x_1}^T\bm{x_1})y_0\bm{x_1} + (x_0\bm{x_1} + x_0\bm{x_1})^T\bm{y_1}\bm{x_1}  \\
%\end{pmatrix}\\
=\,\,\, &
\begin{pmatrix}
(x_0x_0 + \bm{x_1}^T\bm{x_1})(x_0y_0+\bm{x_1}^T\bm{y_1}) + (x_0\bm{x_1}+x_0\bm{x_1})^T(x_0\bm{y_1} + y_0\bm{x_1}) \\
(x_0x_0 + \bm{x_1}^T\bm{x_1})(x_0\bm{y_1} + y_0\bm{x_1}) + (x_0y_0 + \bm{x_1}^T\bm{y_1})(x_0\bm{x_1} + x_0\bm{x_1})
\end{pmatrix}\\
=\,\,\, & \begin{pmatrix}
x_0x_0 + \bm{x_1}^T\bm{x_1} \\
x_0\bm{x_1} + x_0\bm{x_1}
\end{pmatrix}
\circ
\begin{pmatrix}
x_0y_0 + \bm{x_1}^T\bm{y_1} \\
x_0\bm{y_1} + \bm{y_0}\bm{x_1}
\end{pmatrix}\\
=\,\,\, & \Bigr( \Bigr(
\begin{pmatrix} x_0\\ \bm{x_1} \end{pmatrix} \circ
 \begin{pmatrix} x_0\\ \bm{x_1} \end{pmatrix} \Bigr) \circ \Bigr(
\begin{pmatrix} x_0\\ \bm{x_1} \end{pmatrix}  \circ
\begin{pmatrix} y_0\\ \bm{y_1} \end{pmatrix} \Bigr)\Bigr)\\
=\,\,\, & (\bm{x} \circ \bm{x}) \circ ( \bm{x} \circ \bm{y})
\end{split}
\]
従って,結合法則$\bm{x} \circ ( ( \bm{x} \circ \bm{x}) \circ \bm{y} ) = (\bm{x} \circ \bm{x}) \circ ( \bm{x} \circ \bm{y})$が成り立つ.
\item 対称錐$\mathcal{K} = \{ \bm{x} \in \mathbb{R}^{1+n}\,\, |\,\, \exists \bm{y}, \,\, \bm{x} = \bm{y} \circ \bm{y} \}$の条件に対して,
\[
\begin{split}
\bm{x} = \bm{y} \circ \bm{y} \,\,\, & \Longleftrightarrow \,\,\,  \begin{pmatrix} x_0\\ \bm{x_1} \end{pmatrix} = \begin{pmatrix} y_0\\ \bm{y_1} \end{pmatrix} \circ \begin{pmatrix} y_0\\ 
\bm{y_1} \end{pmatrix}\\
& \Longleftrightarrow \,\,\,  \begin{pmatrix} x_0\\ \bm{x_1} \end{pmatrix} = \begin{pmatrix} y_0y_0 + \bm{y_1}^T\bm{y_1} \\ 2y_0\bm{y_1} \end{pmatrix}\\
& \Longleftrightarrow \,\,\, \begin{cases} x_0 = y_0^2 + \bm{y_1}^T\bm{y_1} > 0, \\ \bm{x_1}=2y_0\bm{y_1}  \end{cases}
\end{split}
\]
なので,$\mathcal{K}$の$\bm{x}$に対して,
\[
x_0^2 - \bm{x_1}^T\bm{x_1} = y_0^4 - 2y_0^2\bm{y_1}^T\bm{y_1} + (\bm{y_1}^T\bm{y_1})^2 = (y_0^2 -\bm{y_1}^T\bm{y_1})^2 \ge 0 \]
また,$x_0 > 0$ので, $\|\bm{x_1}\| \le x_0$が成り立つ.従って,$\exists \bm{y}, \,\, \bm{x} = \bm{y} \circ \bm{y} \Rightarrow \|\bm{x_1}\| \le x_0$,逆向きも自明であるので,対称錐$\mathcal{K} = \{ \bm{x} \in \mathbb{R}^{1+n}\,\, |\,\, \exists \bm{y}, \,\, \bm{x} = \bm{y} \circ \bm{y} \}$が$\Bigr\{\begin{pmatrix}
x_0 \\ \bm{x_1}
\end{pmatrix} \,\, \Bigr| \,\,\, \|\bm{x_1}\| \le x_0
 \Bigr\}$と一致する.



\end{enumerate}

\newpage
\section{問題4:解答}
\begin{enumerate}
\item
整数非負問題
\[
X=\{ y \in \mathbb{Z}_+^4 | 4y_1 + 5y_2 + 9y_3 + 12y_4 \le 34 \}
\]
に対して,不等式$y_2+y_3+2y_4 > 6$が妥当不等式であることを示す.$y \in \mathbb{Z}_+^4$ので
\[
\begin{split}
& \,\,\,4y_1 + 5y_2 + 9y_3 + 12y_4 \le 34 \\
\Rightarrow &\,\,\, 5y_2+9y_3 + 12y_4 \le 34\\
\Leftrightarrow &\,\,\, y_2 + 1.8y_3 + 2.4y_4 \le 6.8 \\
\Rightarrow &\,\,\, y_2 + y_3 + 2y_4 \le 6.8 \\
\Rightarrow &\,\,\, y_2 + y_3 + 2y_4 \le 6 \,\,\,\,\,\,(y\in \mathbb{Z}_+^4)\\
\end{split}
\]
$\forall y\in \mathbb{Z}_+^4,\,\, y_2+y_3+2y_4 > 6$が成り立つので,妥当不等式である.
\item
被覆不等式を用いて妥当不等式を構造する.\\
\[
X = \{ X \in \{0,1\}^n\,\,|\,\, \sum_{j=1}^n a_jx_j \le b \}
\]
に対して,
\[
\sum_{j \in C} a_j > b
\]
を満たすとき,被覆不等式とは
\[
\sum_{j\in C} x_j \le |C| -1
\]
である.$X = \{ X \in \{0,1\}^n\,\,|\,\, 7x_1 + 6x_2 + 6x_3 + 4x_4 + 3x_5 \le 14 \}$に対して,すべての$\sum_{j \in C} a_j > b$を満たす$C$を列挙すればよい,なお,$C_i \subset C_j$に対して,$C_i$が被覆の時であるとき$C_j$も被覆であるので,その場合$C_j$を取り除けて極小な被覆集合のみを構造する.
\begin{eqnarray*}
C_1 = \{ 1, 2, 3 \} \,\,&\,\, C_2 = \{ 1, 2, 4 \} \,\,&\,\, C_3 = \{ 1, 2, 5 \}  \\
C_4 = \{ 1, 3, 4 \} \,\,&\,\, C_5 = \{ 1, 3, 5 \} \,\,&\,\, C_6 = \{ 2, 3, 4 \}  \\
C_7 = \{ 2, 3, 5 \} \,\,&\,\, 
\end{eqnarray*}
すべて$C_k$に対応する妥当不等式を作成する.
\begin{eqnarray*}
C_1 : \,\, x_1 + x_2 + x_3 \,\le\, 2 \, &\,\,\,\,\,\,\,\,\,& C_5 : \,\, x_1 + x_3 + x_5 \,\le\, 2 \\
C_2 : \,\, x_1 + x_2 + x_4 \,\le\, 2 \, &\,\,\,\,\,\,\,\,\,& C_6 : \,\, x_2 + x_3 + x_4 \,\le\, 2 \\
C_3 : \,\, x_1 + x_2 + x_5 \,\le\, 2 \, &\,\,\,\,\,\,\,\,\,& C_7 : \,\, x_2 + x_3 + x_5 \,\le\, 2 \\ 
C_4 : \,\, x_1 + x_3 + x_4 \,\le\, 2 \, &\,\,\,&
\end{eqnarray*}
点$\bm{x}=(\frac{1}{7},1,\frac{1}{2},\frac{1}{4},1)$は$C_3$に対応する妥当不等式を満たさないので,カットされている.





\end{enumerate}




\end{document}